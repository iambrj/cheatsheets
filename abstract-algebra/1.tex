\documentclass[titlepage, 12pt]{article}
\usepackage[parfill]{parskip}
\usepackage[nointegrals]{wasysym}
\usepackage{pifont}
\usepackage{amsmath}

\begin{document}

\title{Abstract Algebra: Chapter 1}

\author{Bharathi Ramana Joshi}

\date{\today}

\maketitle

\tableofcontents

\newpage


\section{Basic Axioms}

\begin{enumerate}

  \item A \textit{binary operation} \ding{72} on a set $G$ is a function
    \ding{72}:$G\times G\rightarrow G$.

  \item A binary operation \ding{72} on a set $G$ is \textit{associative} if
    $\forall a, b, c\in G$ we have $a$\ding{72}$(b$\ding{72}$c)$ =
    $(a$\ding{72}$ b)$\ding{72}$ c$.

  \item If \ding{72} is a binary operator on set $G$, we say elements $a$
    and $b$ of $G$ \textit{commute} if $a$\ding{72}$b$  = $b$\ding{72}$a$.
    We say \ding{72} is commutative if for all $a, b\in G$, $a$\ding{72}$b =
    b$\ding{72}$a$.

\end{enumerate}

Definitions

\begin{enumerate}

  \item A group is an ordered pair $(G,$\ding{72}$)$, where $G$ is a set and
    \ding{72} is a binary operator such that

    \begin{enumerate}

      \item \ding{72} is closed under $G$

      \item $a$\ding{72}$(b$\ding{72}$c)$ = $(a$\ding{72}$b)$\ding{72}$c$
        - \ding{72} is associative

      \item $\exists e\in G$ such that $\forall a\in G$ we have
        $a$\ding{72}$e$ = $e$\ding{72}$a$ = a - identity existence

      \item $\forall a\in G$, $\exists a^{-1}\in G$ such that
        $a$\ding{72}$a^{-1}$ = $a^{-1}$\ding{72}$a = e$ - inverse
        existence

    \end{enumerate}

  \item A group $(G,$ \ding{72}) is called an \textit{Abelian/Commutative} if
    it also satisfies the commutative property - $a$\ding{72}$b$ =
    $b$\ding{72}$a$, $\forall a, b\in G$.

  \item For a group $G$ and $x\in G$ the \textit{order} of $x$ is the smallest
    positive integer $n$ such that $x^n = 1$, denoted by $|x|$. $n$ is
    called the order of $x$. If no positive $n$ exists, $x$ is said to be of
    order infinity.

  \item Let $G = {g_1, g_2,\ldots}$ be a finite group with $g_1 = 1$. The
    \textit{multiplication table}/\textit{group table} of $G$ is the
    $n\times n$ matrix whose $i, j$ entry is the group element $g_ig_j$
    (analogue - table of all the distances between pairs of cities in the
    country).

  \item If $(A, \circ)$ and $(B, \bullet)$ are groups, then the new group $A\times B$ is defined as
    \begin{gather}
      A\times B = \{(a, b)\mid a\in A, b\in B\}
    \end{gather}
    and whose operation is defined component wise
    \begin{gather}
      (a_1, b_1)(a_2, b_2) = (a_1\circ a_2, b_1\bullet b_2)
    \end{gather}

  \item If $H$ is a nonempty subset of $G$ such that $\forall h, k\in H, hk$ and
    $h^{-1}\in H$ then $H$ is called the \textbf{subgroup} of $G$

\end{enumerate}

\section{Dihederal groups}

\begin{enumerate}

  \item $D_{2n}$, the \textbf{dihederal} group of order $2n$, is the set of
    symmetries of a regular $n$-gon. Each symmetry $s$ can be described uniquely
    by the corresponding parameter $\sigma$ of $\{1,2,\dots,n\}$ and $st$, for
    $s, t\in D_{2n}$ is defined as applying $t$ and then $s$ (function
    application).

    Define $r$ as a clockwise rotation through the origin by $2\pi/n$ radians
    and $s$ as the reflection about the line of symmetry through vertex 1 and
    the origin. Then

    \begin{enumerate}

      \item $1,r,r^2,\dots,r^{n-1}$ are all distinct and $r^n=1$, so $|r| = n$

      \item $|s| = 2$

      \item $s\neq r^i,\forall i$

      \item $sr^i\neq sr^j, \forall i, j$; so
        \begin{gather*}
          D_{2n} = \{1,r,r^2,\dots,r^{n-1},s,sr,sr^2,\dots,sr^{n-1}\}
        \end{gather*}

      \item $rs=sr^{-1}$

      \item $r^is=sr^{-i}$

    \end{enumerate}

  \item A subset $S\subset G$ with the property that all elements of $G$ can be
    written as a finite product of elements of $S$ is called the
    \textbf{generator} of $G$.

  \item Any equations in a general group $G$ that the generators satisfy are
    called \textbf{relations}.

  \item In general, if some group $G$ is generated by a subset $S$ and there is
    some collection of relations, $R_1,\dots,R_m$, such that any other relation
    among the elements of $S$ can be deduced from these, such generators and
    relations are called a \textbf{presentation} of $G$ and written as
    \begin{gather*}
      G = \langle S\mid R_1, R_2,\dots, R_m\rangle
    \end{gather*}
    for example
    \begin{gather*}
      D_{2n} = \langle r, s\mid r^n = s^2 = 1, rs = sr^{-1}\rangle
    \end{gather*}

\end{enumerate}

\section{Symmetric groups}

\begin{enumerate}

  \item The set of all bijections from a set $\Omega$ onto itself, $S_\Omega$,
    is a group under function composition, $\circ$, called the \textbf{symmetric
    group} on $\Omega$

  \item The special case when $\Omega = \{1,\dots,n\}$ is denoted with $S_n$,
    the \textbf{symmetric group of degree} $n$, with order $n!$

  \item A \textbf{cycle} is a string of integers which represents the element of
    $S_n$ which cyclically permutes these integers and leaves the rest
    unchanged. For instance, the permutation \{12, 13, 3, 1, 11, 9, 5, 10, 6, 4,
    7, 8, 2\} has the cycle decomposition (1 12 8 10 4)(2 13)(5 11 7)(6 9).

  \item The \textit{length} of a cycle is the number of integers that occur in
    it

  \item Two cycles are called \textit{disjoint} if they have no numbers in
    common

  \item Since disjoint cycles permute numbers which lie in disjoint sets, it
    follows that disjoint sets commute

  \item The order of a permutation is the l.c.m of the lengths of the cycles in
    its cycle decomposition

\end{enumerate}

\section{Matrix groups}
\subsection{Field}
\begin{enumerate}
  \item A \textit{field} is a set $F$ together with two commutative binary
    operators + and . on $F$ such that $(F, +)$ is an Abelian group (with
    identity 0) and $(F-\{0\}, 0)$ is also an Abelian group, with the following
    \textit{distributive} law
    \begin{gather*}
      a.(b+c) = (a.b)+(a.c)
    \end{gather*}
  \item For any field $F$ let $F^\times = F-\{0\}$
\end{enumerate}
\subsection{General Linear group}
\begin{gather*}
  GL_n(F) = \{A\mid A \textrm{ is a } n\times n\textrm{ matrix with entries from }
  F\textrm{ and det(}A)\neq 0\}
\end{gather*}
$GL_n(F)$ is a group under matrix multiplication, called \textit{general linear
group of degree n}

Useful facts
\begin{enumerate}

    \item If $F$ is a field and $|F| < \infty$, then $|F| = p^m$ for some prime
        $p$ and integer $m$

    \item If $|F| = q < \infty$, then $|GL_n(F)| = (q^n - 1)(q^n -
        q)\dots(q^n-q^{n - 1})$

\end{enumerate}


\section{Homomorphisms and isomorphisms}
If $(G, \star)$ and $(H, \diamond)$ are groups then $\phi:G\rightarrow H$ such
that
\begin{gather*}
    \phi(x\star y) = \phi(x)\diamond\phi(y), \forall x, y\in G
\end{gather*}
is called a \textbf{homomorphism}.

A homomorphism is called an \textbf{isomorphism} if it is a bijection.

Common techniques to prove two groups, $G$ and $H$, aren't isomorphic

\begin{itemize}

    \item $\mid G\mid \neq \mid H\mid$
        
    \item $G$ is abelian iff $H$ is abelian

    \item $\forall x\in G, \mid x\mid = \mid\phi(x)\mid$
\end{itemize}
\subsection{Miscellaneous}
\begin{itemize}
    \item\textbf{Classification theorem}:What are the objects of a given type, up to some
    equivalence? For instance to prove that if $G$ is an object of with some
    structure and $G$ has some property $P$ then any other similarly structured
    object with property $P$ is isomorphic to $G$.

    \item If $\phi:G\rightarrow H$ is a homomorphism, the \textbf{kernel} of $\phi$ is
    defined as $\{g\in G\mid\phi(g) = 1_H\}$, i.e. set of those elements in $G$ that
    are mapped to the identity element in $H$.

    \item\textbf{Automorphism group}: set of all isomorphisms from $G$ onto $G$
\end{itemize}

\section{Group Actions}
\textbf{Definition}: A \textbf{group action} of a group $G$ on a set $A$ is a
map from $G\times A$ to $A$ satisfying the folllowing properties
\begin{enumerate}

    \item $g_1(g_2.a) = (g_1g_2).a$, for all $g_1, g_2\in G, a\in A$

    \item $1.a = a$, for all $a\in A$

\end{enumerate}

Intuitively, a group action of $G$ on $A$ just means that every element $g\in G$
acts as a permutation on $A$ in a manner consistent with the group operations in
$G$.

Actions of a group $G$ on a set $A$ and the homomorphisms from $G$ into the
symmetric group $S_A$ are in bijective correspondence.

If $G$ acts on a set $A$ and distinct elements of $G$ induce distinct
permutations of $A$, the action is said to be \textbf{faithful}, i.e. associated
permutation representation is injective.

The \textbf{kernel} of the action of $G$ on $A$ is defined to be $\{g\in G\mid
gb = b\ \forall b\in B\}$

\end{document}

