\documentclass[titlepage, 12pt]{article}
\usepackage[parfill]{parskip}
\usepackage{amsmath}
\usepackage{xcolor}
\usepackage{amsfonts}
\usepackage{setspace}
\usepackage{hyperref}
\usepackage{tcolorbox}
\tcbuselibrary{theorems}

\hypersetup{
    colorlinks=true,
    linkcolor=blue,
    filecolor=magenta,
    urlcolor=blue,
}

\newtcbtheorem[]{definition}{Definition}%
{colback=magenta!5,colframe=magenta!100!black,fonttitle=\bfseries}{th}

\newtcbtheorem[]{proposition}{Proposition}%
{colback=cyan!5,colframe=cyan!100!black,fonttitle=\bfseries}{th}

\newtcbtheorem[]{theorem}{Theorem}%
{colback=orange!5,colframe=orange!100!black,fonttitle=\bfseries}{th}

\begin{document}

\begin{titlepage}

	\raggedleft

	\vspace*{\baselineskip}

	{Bharathi Ramana Joshi\\\url{https://github.com/iambrj/notes}}

	\vspace*{0.167\textheight}

	\textbf{\LARGE Notes on}\\[\baselineskip]

	\textbf{\textcolor{teal}{\huge Group Actions}}\\[\baselineskip]

    {\Large \textit{Chapter 4 from Dummit \& Foote, $3^{rd}$ Ed.}}

	\vfill

	\vspace*{3\baselineskip}

\end{titlepage}

\newpage

\begin{proposition}{}{}
    For any group $G$ and any nonempty set $A$ there is a bijection between the
    actions of $G$ on $A$ and the Homomorphisms of $G$ into $S_A$
\end{proposition}

\begin{proposition}{}{}
    If $G$ is a group acting on a nonempty set $A$ then the relation defined by
    \begin{align*}
        a\sim b\textrm{ iff } a = g.b,\textrm{ for some } g\in G
    \end{align*}
    is an equivalence relation. For each $a\in A$ the number of elements in the
    equivalence class containing $a$ is $|G:G_a|$, the index of the stabilizer
    of $a$
\end{proposition}
\textbf{Proof}: Observe that for each $a\in A$, the sets $G/$Stab$(a)$ and
Orbit$(a)$ are isomorphic.

\begin{definition}{}{}
    If $G$ is a group acting on a nonempty set $A$
    \begin{enumerate}
        \item The equivalence class $\{g.a | g\in G\}$ is called the
            \textbf{orbit} of $G$ containing $a$
        \item The action of $G$ on $A$ is called \textbf{transitive} if there
            is only one orbit, i.e., $\forall a, b\in A, \exists g\in G$ such
            that $a = g.b$
    \end{enumerate}
\end{definition}

\begin{theorem}{}{}
    If $G$ acts on its subgroup $H$ by left multiplication on its cosets, then
    \begin{enumerate}
        \item This action is transitive
        \item The stabilizer is $H$
        \item The kernel is $\cap_{x\in G} xHx^{-1}$ and is the largest normal
            subgroup of $G$ contained in $H$
    \end{enumerate}
\end{theorem}
\textbf{Proof:} $\cap_{x\in G} xHx^{-1}$ is a normal subgroup and every other
normal subgroup contained in $H$ is also a subgroup of $\cap_{x\in G} xHx^{-1}$.

\begin{theorem}{Cayley's theorem}{}
    Every group is isomorphic to a subgroup of some symmetric group. If $|G| =
    n$ then $G$ is isomorphic to some subgroup of $S_n$
\end{theorem}
\textbf{Proof:} Consider the group action by $G$ on its identity subgroup $\{e\}$
by coset multiplication. $G$ is isomorphic to its image in $S_n$, by the First
Isomorphism Theorem. The isomorphism is given by $T:G\rightarrow S_G$ where
$T(g) = \sigma_g$

\begin{theorem}{}{}
    If $G$ is a finite group of order $n$ and $p$ is the smallest prime dividing
    $n$, then any subgroup of index $p$ is normal
\end{theorem}
\textbf{Proof:} Let $G$ be the group whose order's smallest prime divisor is $p$
and $H$ a subgroup of $G$ of index $p$. Let $\phi: G\rightarrow S_{G/H}$ be the
permutation representation afforded by the group action and $ker(\phi)$ the
kernel. We now show that $H = ker(\phi)$ as follows: by the First homomorphism
theorem,
\begin{align*}
         &\frac{|G|}{|ker(\phi)|} = |\phi(G)|\\
    \iff &|\phi(G)|\textrm{ divides } |G|
\end{align*}
By Lagrange's theorem, $|\phi(G)|$ divides $|S_{G/H}|$. Now, since it divides them
both, it should also divide their GCD. But observe that, since $p$ is the
\textit{smallest} prime, the GCD itself is $p$. Thus $|\phi(G)|$ is either 1 or
$p$, both of which mean $ker(\phi) = H$ making it normal.

\begin{proposition}{}{}
    Number of conjugates of a subset $S$ of a group $G$ is $|G:N_G(S)|$. In
    particular, the number of conjugates of an element $s$ is $|G:C_G(s)|$.
\end{proposition}

\begin{theorem}{Class Equation}{}
    If $G$ is a finite group and $g_1\dots g_r$ are the representatives of the
    distinct conjugacy classes not contained in $Z(G)$ then
    \begin{gather*}
        |G| = |Z(G)| + \sum_{i = 1}^r |G : C_G(g_i)|
    \end{gather*}
\end{theorem}

\begin{theorem}{}{}
    If $p$ is a prime and $|P| = p^\alpha$ for some $\alpha\geq 1$, then $P$ has
    a nontrivial center, $Z(P)\neq 1$
\end{theorem}
Observe that this implies if $|P| = p^2$ for some prime $p$ then $P$ is abelian
(specifically, either $Z_p\times Z_p$ or $Z_{p^2}$). Proof idea : use class
equation.

\begin{proposition}{}{}
    If $|P| = p^2$ for some prime $p$, then $P$ is abelian (and isomorphic to
    $Z_{p\times p}/Z_{p^2}$)
\end{proposition}
Proof sketch: By previous proposition, it has a nontrivial center making
$P/Z(P)$ cyclic implying $P$ is abelian. If $x$ is a nonidentity element in $P$
then it has order $p$ (if $p^2$ then we are done) and let $y\in P \setminus
\langle x\rangle$. Then since $|\langle x, y\rangle | > |\langle x\rangle |$,
the former must be $P$ itself and since each $x$ and $y$ have order $p$, we have
$\langle x\rangle \times \langle y\rangle\cong Z_{p\times p}$.

\begin{proposition}{}{}
    Conjugations in $S_n$ are just relabelling, i.e. if $\sigma = (a_1\dots
    a_n)$ then for any $\tau$,
    \begin{gather*}
        \tau\sigma\tau^{-1} = (\tau(a_1)\dots\tau(a_n))
    \end{gather*}
\end{proposition}

\begin{definition}{Cycle type}{}
    If $\sigma\in S_n$ is the product of disjoint cycles of lengths
    $n_1,\dots,n_r$ in nondecreasing order, then the integers $n_1,\dots,n_r$
    are called the \textbf{cycle type} of $\sigma$.
\end{definition}

\begin{proposition}{}{}
    Two elements of $S_n$ are conjugates in $S_n$ iff they have the same cycle
    type.
\end{proposition}
\textbf{Proof:} For the forward proof, observe that conjugate permutations have
the same cycle type as conjugation is just relabelling. For the backward proof,
observe that ordering the cycles in nondecreasing length gives us a permutation
mapping elements at corresponding positions in the two decompositions.

Useful observation: Normal subgroups are union of conjugacy classes

\begin{definition}{}{}
    The set of all isomorphisms from a group $G$ to itself is called the
    automorphism group, denoted by Aut$(G)$
\end{definition}

\begin{proposition}{}{}
    If $H\trianglelefteq G$ then $G$ acts by conjugation on $H$ as automorphisms
    of $H$, this action is defined by
    \begin{gather*}
        g.H\mapsto gHg^{-1}
    \end{gather*}
    Furthermore, the permutation representation afforded by this action is a
    Homomorphism of $G$ into Aut($H$) with kernel $C_G(H)$ and $G/C_G(H)$ is
    isomorphic to a subgroup of Aut($H$)
\end{proposition}
A few corollaries:
\begin{enumerate}
    \item $K\leq G\implies\forall g\in G, K\cong gKg^{-1}$
    \item For any subgroup $H\leq G,$ the quotient group $N_G(H)/C_G(H)$ is
        isomorphic to some subgroup of Aut$(G)$.
\end{enumerate}

\begin{definition}{}{}
    If $G$ is a group and $g\in G$, conjugation by $g$ is called an
    \textbf{inner automorphism} of $G$ and the subgroup of Aut$(G)$ consisting
    of all inner automorphisms is denoted by Inn$(G)$
\end{definition}


\begin{definition}{}{}
    A subgroup $H$ of $G$ is called \textbf{characteristic} in $G$, denoted $H$
    char $G$, if every automorphism of $G$ maps $H$ onto itself -
    $\forall\sigma\in$ Aut$(G)$, $\sigma(H) = H$
\end{definition}
\end{document}

