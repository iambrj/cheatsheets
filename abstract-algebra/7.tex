\documentclass[titlepage, 12pt]{article}
\usepackage[parfill]{parskip}
\usepackage{amsmath}
\usepackage{xcolor}
\usepackage{amsfonts}
\usepackage{setspace}
\usepackage{hyperref}
\usepackage{tcolorbox}
\tcbuselibrary{theorems}

\hypersetup{
    colorlinks=true,
    linkcolor=blue,
    filecolor=magenta,
    urlcolor=blue,
}

\newtcbtheorem[]{definition}{Definition}%
{colback=magenta!5,colframe=magenta!100!black,fonttitle=\bfseries}{th}

\newtcbtheorem[]{proposition}{Proposition}%
{colback=cyan!5,colframe=cyan!100!black,fonttitle=\bfseries}{th}

\newtcbtheorem[]{theorem}{Theorem}%
{colback=orange!5,colframe=orange!100!black,fonttitle=\bfseries}{th}

\begin{document}

\begin{titlepage}

	\raggedleft

	\vspace*{\baselineskip}

	{Bharathi Ramana Joshi\\\url{https://github.com/iambrj/notes}}

	\vspace*{0.167\textheight}

	\textbf{\LARGE Notes on}\\[\baselineskip]

	\textbf{\textcolor{teal}{\huge Introduction to Rings}}\\[\baselineskip]

    {\Large \textit{Chapter 7 from Dummit \& Foote, $3^{rd}$ Ed.}}

	\vfill

	\vspace*{3\baselineskip}

\end{titlepage}

\newpage

\begin{definition}{Ring}{}
    A \textbf{ring} $R$ is a set with two binary operations $+$ and $\times$
    such that
    \begin{enumerate}
        \item $(R, +)$ is an abelian group
        \item $\times$ is associative : $(a\times b)\times c = a\times (b\times
            c)$
        \item The following distributive laws hold
            \begin{enumerate}
                \item $(a+b)\times c = (a\times c) + (b\times c)$
                \item $a\times (b+c) = (a\times b) + (a\times c)$
            \end{enumerate}
    \end{enumerate}
\end{definition}

A ring is commutative if multiplication is commutative

A ring is said to have an identity if $\exists 1\in R$ such that
\begin{gather*}
    1\times a = a\times 1 = a, \forall a\in R
\end{gather*}

\begin{definition}{Division ring}{}
    A ring $R$ with an identity element 1 ($\neq 0$) is called a
    \textbf{division ring/skew field} if $\forall a\neq 0\in R, \exists b\in R$
    such that $ab=ba=1$
\end{definition}

A commutative division ring is called a \textbf{field}

\begin{proposition}{}{}
    If $R$ is a ring then
    \begin{enumerate}
        \item $0a=a0=0, \forall a\in R$
        \item $(-a)b=a(-b)=-(ab), \forall a,b\in R$
        \item $(-a)(-b) = ab,\forall a,b\in R$
        \item If $R$ has identity 1, then it is unique and $-a = (-1)a$
    \end{enumerate}
\end{proposition}

\begin{definition}{Zero divisor, unit}{}
    \begin{enumerate}
        \item A nonzero element $a\in R$ is called a \textbf{zero divisor} if
        $\exists b\neq 0\in R$ such that $ab=0$ or $ba=0$
        \item If $R$ has an identity $1\neq 0$, then $u\in R$ is called
        \textbf{unit} in $R$ if $\exists v\in R$ such that $uv=vu=1$. Set of
        units is denoted by $R^\times$
    \end{enumerate}
\end{definition}
 Note:
\begin{enumerate}
    \item The units in a ring $R$ form a group under multiplication
    \item A zero divisor can never be a unit
\end{enumerate}

\begin{definition}{Integral domain}{}
    A commutative ring with identity $1\neq 0$ is called an \textbf{integral
    domain} if it has no zero divisors
\end{definition}

\begin{proposition}{}{}
    If $a,b,c\in R$ and $a$ not a zero divisor, then $ab=ac\implies a = 0 \lor b =
    c$, equivalent $ab=ac\implies a = 0\lor b = c$
\end{proposition}

Note that any finite integral domain is a field

\begin{definition}{Subring}{}
    A \textbf{subring} of a ring $R$ is a subgroup of $R$ closed under
    multiplication
\end{definition}

\end{document}

