\documentclass[titlepage, 12pt]{book}
\usepackage[parfill]{parskip}
\usepackage{amsmath}
\usepackage{amssymb}
\usepackage{pifont}
\usepackage{amsfonts}
\usepackage{xcolor}
\usepackage{setspace}
\usepackage{hyperref}
\usepackage{tcolorbox}
\usepackage{epigraph}

\tcbuselibrary{theorems}

\hypersetup{
    colorlinks=true,
    linkcolor=blue,
    filecolor=magenta,
    urlcolor=blue,
}

\newtcbtheorem[]{definition}{Definition}%
{colback=magenta!5,colframe=magenta!100!black,fonttitle=\bfseries}{th}

\newtcbtheorem[]{proposition}{Proposition}%
{colback=cyan!5,colframe=cyan!100!black,fonttitle=\bfseries}{th}

\newtcbtheorem[]{theorem}{Theorem}%
{colback=orange!5,colframe=orange!100!black,fonttitle=\bfseries}{th}

\begin{document}

\begin{titlepage} % Suppresses headers and footers on the title page

	\raggedleft% Right align everything

	\vspace*{\baselineskip} % Whitespace at the top of the page

	{Bharathi Ramana Joshi\\\url{https://github.com/iambrj/notes}} % Author name

	\vspace*{0.167\textheight} % Whitespace before the title

	\textbf{\LARGE Notes on}\\[\baselineskip] % First title line

	\textbf{\textcolor{teal}{\huge Abstract Algebra}}\\[\baselineskip] % Main title line which draws the focus of the reader

    {\Large \textit{Based on Dummit \& Foote, $3^{rd}$ Ed.}} % Subtitle

	\vfill % Whitespace between the titles and the publisher

	\vspace*{3\baselineskip} % Whitespace at the bottom of the page

\end{titlepage}

\tableofcontents

\chapter{Introduction to Groups}
\begin{definition}{Group}{}
  A group is an ordered pair $(G,$\ding{72}$)$, where $G$ is a set and
    \ding{72} is a binary operator such that

    \begin{enumerate}

      \item \ding{72} is closed under $G$

      \item $a$\ding{72}$(b$\ding{72}$c)$ = $(a$\ding{72}$b)$\ding{72}$c$
        - \ding{72} is associative

      \item $\exists e\in G$ such that $\forall a\in G$ we have
        $a$\ding{72}$e$ = $e$\ding{72}$a$ = a - identity existence

      \item $\forall a\in G$, $\exists a^{-1}\in G$ such that
        $a$\ding{72}$a^{-1}$ = $a^{-1}$\ding{72}$a = e$ - inverse
        existence

    \end{enumerate}
\end{definition}

\textbf{Single axiom:}
\begin{gather*}
y \times (z \times (((w \times w^{-1}) \times (x \times z)^{-1}) \times y))^{-1} = x
\end{gather*}

If $G$ is a group under the operation \ding{72}, then
\begin{enumerate}
    \item the identity element of $G$ is unique
    \item $\forall a\in G$, $a^{-1}$ is unique
    \item $(a^{-1})^{-1} = a$, $\forall a\in G$
    \item $(a$\ding{72}$b)^{-1} = b^{-1}$\ding{72}$a^{-1}$
\end{enumerate}

\begin{definition}{Abelian group}{}
    A group $(G,$ \ding{72}) is called an \textit{Abelian/Commutative} if
    it also satisfies the commutative property - $a$\ding{72}$b$ =
    $b$\ding{72}$a$, $\forall a, b\in G$.
\end{definition}

\begin{definition}{Order}{}
    For a group $G$ and $x\in G$ the \textit{order} of $x$ is the smallest
    positive integer $n$ such that $x^n = 1$, denoted by $|x|$. $n$ is
    called the order of $x$. If no positive $n$ exists, $x$ is said to be of
    order infinity.
\end{definition}

\begin{definition}{Multiplication/Group table}{}
  Let $G = {g_1, g_2,\ldots}$ be a finite group with $g_1 = 1$. The
    \textit{multiplication table}/\textit{group table} of $G$ is the
    $n\times n$ matrix whose $i, j$ entry is the group element $g_ig_j$
    (analogue - table of all the distances between pairs of cities in the
    country).
\end{definition}

\begin{definition}{Cross product of groups}{}
  If $(A, \circ)$ and $(B, \bullet)$ are groups, then the new group $A\times B$ is defined as
    \begin{gather*}
      A\times B = \{(a, b)\mid a\in A, b\in B\}
    \end{gather*}
    and whose operation is defined component wise
    \begin{gather*}
      (a_1, b_1)(a_2, b_2) = (a_1\circ a_2, b_1\bullet b_2)
    \end{gather*}
\end{definition}

\begin{definition}{Subgroup}{}
  If $H$ is a nonempty subset of $G$ such that $\forall h, k\in H, hk$ and
    $h^{-1}\in H$ then $H$ is called the \textbf{subgroup} of $G$
\end{definition}

\textbf{Subgroup test:} $H\leq G\iff\forall x, y\in H, xy^{-1}\in H$

\begin{definition}{Dihederal group}{}
    $D_{2n}$, the \textbf{dihederal} group of order $2n$, is the set of
    symmetries of a regular $n$-gon. Each symmetry $s$ can be described uniquely
    by the corresponding parameter $\sigma$ of $\{1,2,\dots,n\}$ and $st$, for
    $s, t\in D_{2n}$ is defined as applying $t$ and then $s$ (function
    application).

    Define $r$ as a clockwise rotation through the origin by $2\pi/n$ radians
    and $s$ as the reflection about the line of symmetry through vertex 1 and
    the origin. Then

  \begin{enumerate}
      \item $1,r,r^2,\dots,r^{n-1}$ are all distinct and $r^n=1$, so $|r| = n$

      \item $|s| = 2$

      \item $s\neq r^i,\forall i$

      \item $sr^i\neq sr^j, \forall i, j$; so
        \begin{gather*}
          D_{2n} = \{1,r,r^2,\dots,r^{n-1},s,sr,sr^2,\dots,sr^{n-1}\}
        \end{gather*}

      \item $rs=sr^{-1}$

      \item $r^is=sr^{-i}$
  \end{enumerate}

\end{definition}

\begin{definition}{Generator}{}
  A subset $S\subset G$ with the property that all elements of $G$ can be
    written as a finite product of elements of $S$ is called the
    \textbf{generator} of $G$.
\end{definition}

\begin{definition}{Relations}{}
  Any equations in a general group $G$ that the generators satisfy are
    called \textbf{relations}.
\end{definition}

\begin{definition}{Presentation of a group}{}
  In general, if some group $G$ is generated by a subset $S$ and there is
    some collection of relations, $R_1,\dots,R_m$, such that any other relation
    among the elements of $S$ can be deduced from these, such generators and
    relations are called a \textbf{presentation} of $G$ and written as
    \begin{gather*}
      G = \langle S\mid R_1, R_2,\dots, R_m\rangle
    \end{gather*}
    for example
    \begin{gather*}
      D_{2n} = \langle r, s\mid r^n = s^2 = 1, rs = sr^{-1}\rangle
    \end{gather*}
\end{definition}

\begin{definition}{Symmetric group}{}
  The set of all bijections from a set $\Omega$ onto itself, $S_\Omega$,
    is a group under function composition, $\circ$, called the \textbf{symmetric
    group} on $\Omega$

  The special case when $\Omega = \{1,\dots,n\}$ is denoted with $S_n$,
    the \textbf{symmetric group of degree} $n$, with order $n$!
\end{definition}

\begin{definition}{Cycle}{}
  A \textbf{cycle} is a string of integers which represents the element of
    $S_n$ which cyclically permutes these integers and leaves the rest
    unchanged. For instance, the permutation \{12, 13, 3, 1, 11, 9, 5, 10, 6, 4,
    7, 8, 2\} has the cycle decomposition (1 12 8 10 4)(2 13)(5 11 7)(6 9).
\end{definition}

\begin{definition}{Length of a cycle}{}
  The \textit{length} of a cycle is the number of integers that occur in
    it
\end{definition}

\begin{definition}{Disjoint cycles}{}
  Two cycles are called \textit{disjoint} if they have no numbers in
    common
\end{definition}

\begin{proposition}{}{}
Since disjoint cycles permute numbers which lie in disjoint sets, it
follows that disjoint cycles commute
\end{proposition}

\begin{proposition}{}{}
The order of a permutation is the l.c.m of the lengths of the cycles in
its cycle decomposition
\end{proposition}


\begin{definition}{Field}{}
    A \textit{field} is a set $F$ together with two commutative binary
    operators + and $\times$ on $F$ such that $(F, +)$ is an Abelian group (with
    identity 0) and $(F-\{0\}, \times)$ is also an Abelian group, with the following
    \textit{distributive} law
    \begin{gather*}
        a.(b+c) = (a.b)+(a.c)
    \end{gather*}
\end{definition}
As a matter of convention, for any field $F$ let $F^\times = F-\{0\}$

\begin{definition}{General Linear group}{}
    Let
\begin{gather*}
GL_n(F) = \{A\mid A \textrm{ is a } n\times n\textrm{ matrix with entries from }
F\textrm{ and det(}A)\neq 0\}
\end{gather*}
$GL_n(F)$ is a group under matrix multiplication, called \textbf{general linear
group of degree n}
\end{definition}

Useful facts
\begin{enumerate}

    \item If $F$ is a field and $|F| < \infty$, then $|F| = p^m$ for some prime
        $p$ and integer $m$

    \item If $|F| = q < \infty$, then $|GL_n(F)| = (q^n - 1)(q^n -
        q)\dots(q^n-q^{n - 1})$

\end{enumerate}


\begin{definition}{Homomorphism}{}
If $(G, \star)$ and $(H, \diamond)$ are groups, then $\phi:G\rightarrow H$ such
that
\begin{gather*}
    \phi(x\star y) = \phi(x)\diamond\phi(y), \forall x, y\in G
\end{gather*}
is called a \textbf{homomorphism}.
\end{definition}

\begin{definition}{Isomorphism}{}
A homomorphism is called an \textbf{isomorphism} if it is a bijection.
\end{definition}

Common techniques to prove two groups, $G$ and $H$, aren't isomorphic is to
disprove the following

\begin{itemize}

    \item $\mid G\mid = \mid H\mid$
        
    \item $G$ is abelian iff $H$ is abelian

    \item $\forall x\in G, \mid x\mid = \mid\phi(x)\mid$
\end{itemize}

\begin{definition}{Classification theorem}{}
{Classification theorem}:What are the objects of a given type, up to some
equivalence? For instance to prove that if $G$ is an object of with some
structure and $G$ has some property $P$ then any other similarly structured
object with property $P$ is isomorphic to $G$.
\end{definition}

\begin{definition}{Kernel of a homomorphism}{}
$\phi:G\rightarrow H$ is a homomorphism, the \textbf{kernel} of $\phi$ is
defined as $\{g\in G\mid\phi(g) = 1_H\}$, i.e. set of those elements in $G$ that
are mapped to the identity element in $H$.
\end{definition}

\begin{definition}{Automorphism group}{}
\textbf{Automorphism group} is set of all isomorphisms from $G$ onto $G$
equipped with function composition
\end{definition}

\begin{definition}{Group action}{}
A \textbf{group action} of a group $G$ on a set $A$ is a
map from $G\times A$ to $A$ satisfying the folllowing properties
\begin{enumerate}

    \item $g_1(g_2.a) = (g_1g_2).a$, for all $g_1, g_2\in G, a\in A$

    \item $1.a = a$, for all $a\in A$

\end{enumerate}
\end{definition}

Intuitively, a group action of $G$ on $A$ just means that every element $g\in G$
acts as a permutation on $A$ in a manner consistent with the group operations in
$G$. Proof: let $g\in G$ be arbitrary but fixed element, then $gA\subseteq
A$ by definition and for arbitrary but fixed $a\in A$,
\begin{align*}
    a &= 1 . a \textrm{ (by 1)}\\
      &= (gg^{-1}) . a\\
      &= g . (g^{-1} . a) \textrm{ (by 2)}\\
      &\in gA  \textrm{(by forward direction proof)}
\end{align*}
thus proving $A\subseteq gA$. Since both belong to each other, both are equal.

\begin{proposition}{}{}
Actions of a group $G$ on a set $A$ and the homomorphisms from $G$ into the
symmetric group $S_A$ are in bijective correspondence.
\end{proposition}
Proof: for any homomorphism $\phi: G\rightarrow S_A$, the map $g . a =
\phi(g)(a), \forall g\in G, a\in A$ is a group action.

\begin{definition}{Faithful action}{}
If $G$ acts on a set $A$ and distinct elements of $G$ induce distinct
permutations of $A$, the action is said to be \textbf{faithful}, i.e. associated
permutation representation is injective. Also, faithful actions have the trivial
kernel.
\end{definition}

\begin{definition}{Kernel}{}
The \textbf{kernel} of the action of $G$ on $A$ is defined to be $\{g\in G\mid
gb = b\ \forall b\in B\}$, i.e. all the elements in $G$ that fix \textit{all}
the elements of $A$. The kernel is normal subgroup of $G$.
\end{definition}

Note: Two elements in the group induce the same permutation iff they are in the
same coset of the kernel

\begin{definition}{Stabilizer}{}
If $G$ is a group acting on a set $A$, then for some fixed $a\in A$, the
subgroup $\{g\in G\mid ga=a\}$ is called the \textbf{stabilizer} of $a$ in $G$.
\end{definition}

Intuitively, it is the ssubgroup of those elements of $G$ that fix $a$.

\begin{definition}{Left regular action}{}
    For any group $G$ and $A = G$, the map from $G\times A$ to $A$ defined by
    $g.a = ga$ is called the \textbf{left regular action} of $G$ on itself.
\end{definition}

\begin{definition}{Conjugation}{}
    If $G$ is any group then the action defined by $g.a = gag^{-1}$ is called
    the \textbf{conjugation}.
\end{definition}


\begin{definition}{Orbit}{}
    If $G$ is a group acting on a set $X$, the \textbf{orbit} of an element
    $x\in X$ is defined as $\{y\in X\mid\exists g\in G$ such that $y = g$ \ding{72}
    $x\}$
\end{definition}
Intuitively, the orbit of an element is the set of all those elements that can
be reached using the group action.

\chapter{Subgroups}
\begin{definition}{Subgroup}{}
    If $G$ is a group and $H$ is a subset of $G$ such that
    \begin{enumerate}
        \item $H$ is nonempty
        \item $H$ is closed under $G$'s group operator (i.e. $x, y\in H\implies
            xy\in H$)
        \item $H$ is closed under inverses (i.e. $x\in H \implies x^{-1} \in H$)
    \end{enumerate}
    then $H$ is called a \textbf{subgroup} of $G$.
\end{definition}

\textbf{Subgroup criterion}: A subset $H$ of a group $G$ is a subgroup iff
\begin{enumerate}
    \item $H\neq\phi$
    \item $\forall x, y\in H, xy^{-1}\in H$
\end{enumerate}
Furthermore, if $H$ is finite it sufficies to check that $H$ is nonempty and
closed under the group operation.

\begin{definition}{Centralizer}{}
    $\forall A\subseteq G, C_G(A) = \{g\in G| gag^{-1} = a,\forall a\in A\}$ is
    a subset of $G$ and is called the \textbf{centralizer} of $A$ in $G$. Since
    $gag^{-1} = a$ iff $ga = ag$, $C_G(A)$ is the set of elements of $G$ which
    commute with every element of $A$.
\end{definition}
Intuitively, the centralizer of $A$ measures how inside the center $Z(G)$ $A$
is.

\begin{definition}{Center}{}
    The subset $Z = \{g\in G| gx = xg,\forall x\in G\}$ of $G$ is called the
    \textbf{center} of $G$.
\end{definition}
The center of a set is just the centralizer with $A = G$

\begin{definition}{Normalizer}{}
    If $gAg^{-1} = \{gag^{-1}| a\in A\}$, the \textbf{normalizer}  of $A
    (\subseteq G)$ in $G$ is defined as $N_G(A) = \{g\in G| gAg^{-1} = A \}$
\end{definition}
Intuitively, the normalizer of a set $A$ measures how normal the set $A$ is.


On a side note, observe that if $g\in C_G(A)$ then $gag^{-1} = a\in A$ for all
$a\in A$ so $C_G(A)\le N_G(A)$

\begin{definition}{Stabilizer}{}
    If $G$ is a group an $S$ is a set on which $G$ acts, then for any fixed
    $s\in S$ the \textbf{stabilizer} of $s$ is defined as $\{g\in G| g.s=s\}$
\end{definition}
Note that centralizer, normalizer and stabilizer are all subgroups of $G$.

\begin{definition}{Kernel of an action}{}
    The kernel of the action of $G$ on $S$ is defined as $\{g\in G| g.s = s,
    \forall s\in S\}$
\end{definition}
\begin{definition}{Cyclic group}{}
    A group $H$ is \textbf{cyclic} if $H$ can be generated by a single element,
    i.e. $\exists x\in H$ such that $H = \{x^n| n\in \mathbb{Z}\}$. This is denoted
    using $H = \langle x\rangle$ and $x$ is said to generate $H$.
\end{definition}
\begin{proposition}{}{}
    Cyclic groups are abelian
\end{proposition}
\begin{proposition}{}{}
    If $H = \langle x\rangle$, then $| H| = | x|$
\end{proposition}
\begin{proposition}{}{}
    For any element $x\in G$ of an arbitrary group $G$, and $m, n\in
        \mathbb{Z}$,
        if $x^n = 1$ and $x^m = 1$, then $x^{(m, n)} = 1$. In particular, if
        $x^m = 1$ for some $m\in \mathbb{Z}$, then $|x|$ divides $m$.
\end{proposition}
Intuitively, $x^m = 1$ iff $m$ is a multiple of $|x|$
\begin{proposition}{}{}
    Any two cyclic groups of same order are isomorphic
\end{proposition}
\begin{proposition}{}{}
    For any element $x\in G$ of an arbitrary group $G$, and $a\in
        \mathbb{Z} - \{0\}$
        \begin{enumerate}
            \item If $|x| = \infty$, then $|x^a| = \infty$.
            \item If $|x| = n < \infty$, then $|x^a| = \frac{n}{(n, a)}$.
            \item In particular, if $|x| = n < \infty$ and $a$ is a positive integer
                dividing $n$, then $|x^a| = \frac{n}{a}$.
        \end{enumerate}
\end{proposition}
Proof sketch (2) - The proof is broken down into two parts
\begin{enumerate}
    \item $|x^a|\mid\frac{n}{(n, a)}$ : for any $y$ in any finite group, only
        order multiple powers $m = k\times |y|$ can make $y^m = e$. Since
        $(x^a)^{\frac{n}{(n, a)}} = 1$,
        $\frac{n}{(n, a)}$ must be a multiple of $|x^a|$ 
    \item $\frac{n}{(n, a)}\mid|x^a|$
        \begin{align*}
                 & n\mid a\times |x^a|\\
            \iff & (n, a)\times\frac{n}{(n, a)}\mid (n, a)\times\frac{a}{(n, a)}\times|x^a|\\
            \iff & \frac{n}{(n, a)}\mid \frac{a}{(n, a)}\times|x^a|\\
            \iff & \frac{n}{(n, a)}\mid|x^a| (\textrm{since coprime})
        \end{align*}
\end{enumerate}

\begin{proposition}{}{}
    If $H = \langle x\rangle$ then
        \begin{enumerate}
            \item If $|x| = \infty$ then $H = \langle x^a\rangle$ iff $a =
                \pm 1$
            \item If $|x| = n < \infty$ then $H = \langle x^a\rangle$ iff $(a, n) =
                1$. In particular the number of generators of $H$ is $\phi(n)$
        \end{enumerate}
\end{proposition}
\begin{proposition}{}{}
    If $H = \langle x\rangle$ is a cyclic group then
        \begin{enumerate}
            \item Every subgroup of $H$ is cyclic, more precisely if $K\le H$,
                then either $K = \{1\}$ or $K = \{x^d\}$, where $d$ is the
                smallest positive integer such that $x^d\in K$.
            \item If $|H| = \infty$ then for all $a, b\in\mathbb{Z}$ such that
                $a\neq b$, $\langle x^a\rangle\neq\langle x^b\rangle$ and for
                every $m\in\mathbb{Z}$, $\langle x^m\rangle = \langle
                x^{|m|}\rangle$, i.e.\ nontrivial subgroups of $H$ are in a
                bijective correspondence with the natural numbers.
            \item If $|H| = n < \infty$ then for all $a\in\mathbb{Z}$ such that
                $a|n$, $\exists$ unique subgroup $H$ of order $a$ (which is the
                cyclic group $\langle x^{\frac{n}{a}}\rangle$), and for every
                $m\in\mathbb{Z}$, $\langle x^m\rangle = \langle x^{(n,
                m)}\rangle$, i.e.\ subgroups of $H$ are in a bijective
                correspondence with the positive divisors of $n$.
        \end{enumerate}
\end{proposition}
\textbf{Proof sketch}: Consider a nontrivial subgroup $K\leq H$ and let\newline
$\mathcal{P} = \{b\;|\;b\in\mathbb{Z^+}\textrm{ and } x^b\in K\}$. By the well
ordering principle, $\mathcal{P}$ has a minimum element, call it $d$. Now,
$\langle x^d\rangle\leq K$. Also $K\leq\langle x^d\rangle$ since
\begin{align*}
    \forall a\in\mathcal{P}, a &= dq+r\textrm{ , for appropriate $r\in [0, d)$}\\
    \implies x^r &= x^{a-dq} = x^a (x^d)^{-q}\in K.\\
    \because d\textrm{ is minimum }, r &= 0\\
    \implies x^a &= (x^d)^q \in \langle x^d\rangle
\end{align*}

\begin{proposition}{}{}
    If $\mathcal{A}$ is any nonempty collection of subgroups of $G$, then
        the intersection of all members of $\mathcal{A}$ is also a subgroup of
        $G$
\end{proposition}
\begin{definition}{Generated subgroup}{}
    If $A$ is any subset of the group $G$, then the \textbf{subgroup of
        $G$ generated by $A$} is defined as
        \begin{gather*}
            \langle A\rangle = \bigcap_{A\subseteq H,\forall H\le G}H
        \end{gather*}
    $\bar{A} = \langle A\rangle$\\
        where,
        \begin{gather*}
            \bar{A} = \{a_1^{\epsilon_1}a_2^{\epsilon_2}\dots
                a_n^{\epsilon_n}|n\in\mathbb{Z}, n\ge 0\textrm{ and } a_i\in A,
            \epsilon_i = \pm 1, \forall i\}
        \end{gather*}
        i.e. it is the set of all finite products (called \textbf{words}) of
        elements of $A$ and inverses of elements of $A$
\end{definition}
\begin{definition}{Lattice}{}
    The \textbf{lattice} of subgroups of a given finite group $G$ is constructed
    by plotting all subgroups of $G$ starting at the bottom with 1, ending at
    the top with $G$ and with subgroups of larger order positioned higher on the
    page than those of smaller order. There is a line upward from $A$ to $B$ if
    $A\le B$ and there are no subgroups properly between $A$ and $B$.
\end{definition}
Thus for any pair of subgroups $H$ and $K$ of $G$ the unique smallest subgroup
which contains both of them, $\langle H, K\rangle$ (called the \textbf{join} of
$H$ and $K$) is first common ancestor.
\begin{proposition}{}{}
    Isomorphic groups have same lattices.
\end{proposition}
However, nonisomorphic groups may also have identical lattices (i.e. don't fall
for the converse error).

\chapter{Quotient Groups and Homomorphisms}

\begin{definition}{Kernel}{}
    If $\phi$ is a homomorphism $\phi:G\rightarrow H$, the \textbf{kernel} of
    $\phi$ is the set
    \begin{gather*}
        \{g\in G\mid\phi(g) = 1\}
    \end{gather*}
    and is denoted using ker$\phi$.
\end{definition}
If $G$ and $H$ are groups and $\phi:G\rightarrow H$ is a homomorphism
then
\begin{enumerate}
    \item$\phi(1_G) = 1_H$.
    \item$\phi(g^{-1}) = \phi(g)^{-1}$, $\forall g\in G$.
    \item$\phi(g^n) = \phi(g)^n$, $\forall n\in\mathbb{Z}$.
    \item ker$\phi$ is a subgroup of $G$.
    \item im$(\phi)$, the image of $G$ under $\phi$ is a subgroup of $H$.
\end{enumerate}

\begin{definition}{Quotient/Factor group}{}
    If $\phi:G\rightarrow H$ is a homomorphism with kernel $K$, the
    \textbf{quotient group} or \textbf{factor group}, $G/K$ (read $G$ modulo $K$
    or simply $G$ mod $K$) is the group whose elements are the fibers of $\phi$
    with group operation defined as follows: if $X$ is the fiber above $a$ and
    $Y$ is the fiber above $b$ then the product of $X$ with $Y$ is defined to be
    the fiber above the product $ab$.
\end{definition}
In other words, it is the group whose elements are subsets of $G$ being mapped
to the same element in $H$ and group operation is as defined above.

\begin{proposition}{}{}
    If $\phi:G\rightarrow H$ is a homomorphism with kernel $K$ and $X\in G/K$
    such that $X = \phi^{-1}(a)$ for some $a$ then \begin{gather*} \forall u\in X,\;X = \{uk | k\in K\} = \{ku | k\in K\}.
    \end{gather*}
\end{proposition}
\begin{definition}{Cosets}{}
    $\forall N\le G$ and $\forall g\in G$
    \begin{gather*}
       gN = \{gn | n\in N\}\\
       Ng = \{ng | n\in N\}
    \end{gather*}
    are defined as the \textbf{left coset} and \textbf{right coset} of $N$ in
    $G$ respectively. Any element of a coset is called a \textbf{representative}
    for the coset.
\end{definition}
\begin{theorem}{}{}
    If $G$ is a group and $K$ is the kernel of some homomorphism from $G$ to
    another group, then the set whose elements are the left cosets of $K$ in $G$
    with the operation defined by
    \begin{gather*}
        uK\circ vK = (uv)K
    \end{gather*}
    forms the group $G/K$.
\end{theorem}
\begin{proposition}{}{}
    If $N$ is \textit{any} subgroup of the group $G$, the set of left (right) cosets of $N$ in
    $G$ form a partition of $G$. Furthermore, $\forall u, v\in G, uN = vN$ iff
    $v^{-1}u\in N$ and in particular, $uN = vN$ iff $u$ and $v$ are
    representatives of the same coset.
\end{proposition}
\textbf{Proof sketch:} $1\in N$ therefore $\bigcup\limits_{g\in G} gN = G$ and
$uN\cap vN = \phi$ for $u$ and $v$ from different cosets since otherwise
\begin{align*}
    un_1 &= vn_2\\
    \iff u(n_1n_2^{-1}) &= v\implies v\in uN\;\textrm(contradiction!)
\end{align*}

\begin{proposition}{}{}
    If $G$ is a group, then $\forall N\le G$,
    \begin{enumerate}
        \item The operation on the set of left (right) cosets of $N$ in $G$ defined as
            \begin{gather*}
                uN\cdot vN = (uv)N
            \end{gather*}
            is well defined iff $gng^{-1}\in N,\;\forall g\in G$ and $\forall
            n\in N$.
        \item If the above operation is well defined, then the set of left (right)
            cosets of $N$ in $G$ form a group whose identity is $1N$ and
            $(gN)^{-1} = g^{-1}N$. This group is denoted by $G/N$.
    \end{enumerate}
\end{proposition}
\textbf{Proof sketch:} For the forward proof, we want to prove $gNg^{-1} = N$
from $uN.vN=(uv)N$. Multiplying both sides by $N$,
\begin{align*}
    &gNg^{-1} = N\\
    \iff &gN^{-1}N = N\\
    \iff &N = N
\end{align*}
For the backward proof we just show $uN.vN = (uv)N$ from $gN = Ng$.

Corollary: a subgroup $N$ of $G$ is normal iff it is the kernel of some
homomorphism.

\begin{definition}{Conjugate}{}
    The element $gng^{-1}$ is called the \textbf{conjugate} of $n\in N$ by $g$
    and the set $gNg^{-1} = \{gng^{-1}\;|\;n\in N\}$ is called the
    \textbf{conjugate} of $N$ by $g$. An element $g\in G$ is said to
    \textbf{normalize} $N$ if $gNg^{-1} = N$. A subgroup $N$ of $G$ is called
    \textbf{normal} if $gNg^{-1} = N,\;\forall g\in G$. If $N$ is a normal
    subgroup of $G$, it is written as $N\trianglelefteq G$.
\end{definition}
\begin{theorem}{}{}
    $\forall N\le G$, the following are equivalent
    \begin{enumerate}
        \item $N\trianglelefteq G$.
        \item $N_G(N) = G$.
        \item $gN = Ng,\;\forall g\in G$.
        \item The operation on left (right) cosets defined above makes it a
            group.
        \item $gNg^{-1}\subseteq N,\;\forall g\in G$.
        \item $N$ is the kernel of some homomorphism from $G$
    \end{enumerate}
\end{theorem}
Tips to minimize computations necessary to determine whether a subgroup $N$ of
$G$ is normal
\begin{enumerate}
    \item Avoid as much as possible the computation of all the conjugates
        $gng^{-1}$, $\forall n\in N$ and $\forall g\in G$.
    \item If set of generators of $N$ are known, it suffices to check that all
        conjugates of these generators lie in $N$ to prove $N$ is normal.
    \item If set of generators of $G$ are known, it suffices to check that these
        generators normalize $N$.
    \item If generators for both $N$ and $G$ are known, and $N$ is finite, it
        suffices to check that the conjugates of a set of generators for $N$ by
        a set of generators for $G$ are again elements of $N$.
\end{enumerate}
\begin{definition}{Natural projection/homomorphism}{}
    If $N\trianglelefteq G$ then then homomorphism defined by $\pi:G\rightarrow
    G/N$ defined by $\pi(g) = gN$ is called the \textbf{natural
    projection/homomorphism} of $G$ onto $G/N$. If $\overline H\le G/N$ is a
    subgroup of $G/N$, the \textbf{complete preimage} of $\overline H$ in $G$ is
    the preimage of $\overline H$ under the natural projection homomorphism.
\end{definition}
\begin{theorem}{Lagrange's Theorem}{}
    If $G$ is a finite group and $H$ is a subgroup of $G$, then the order of $H$
    divides the order of $G$ and the number of left cosets of $H$ in $G$ is
    $|G|/|H|$.
\end{theorem}
\textbf{Proof sketch:} By Proposition 2, the left (right) cosets of $H$ form a
partition of $G$. Furthermore, $\forall g\in G$, $|gH| = |H|$ since otherwise
$gh_1$ = $gh_2$ (for distinct $h1, h2\in H$) $\implies h_1 = h_2$.
\begin{definition}{Index}{}
    If $G$ is a group and $H\le G$, the number of left cosets of $H$ in $G$ is
    called the \textbf{index} $H$ in $G$ and is denoted by $|G:H|$.
\end{definition}
A few corollaries that follow
\begin{enumerate}
    \item If $G$ is a finite group, then $\forall x\in G$ $|x|\mid |G|$ and in
        particular $x^{|G|} = 1$.
    \item If $G$ is a group of prime order $p$, then $G$ is cyclic (and $G\cong
        Z_p$)
\end{enumerate}
\begin{theorem}{Cauchy's Theorem}{}
    If $G$ is a finite group and $p$ is a prime dividing $|G|$, then $G$ has an
    element of order $p$.
\end{theorem}
\textbf{Proof sketch - 1}: case wise induction on $G$\\
\textbf{Proof sketch - 2}: Consider the set
\begin{gather*}
    S = \{(g_1\dots g_p) | g_1\dots g_p = e\}
\end{gather*}
Then
\begin{enumerate}
    \item $|S| = |G|^{p-1}$
    \item $S$ can be partitioned into equivalence classes based on cyclic
        permutations
    \item Each equivalence class either has 1 or $p$ elements, since otherwise
        \begin{gather}
            (g_1,\dots,g_p)>>k = (g_1,\dots,g_p)\\
            (g_1,\dots,g_p)>>p = (g_1,\dots,g_p)
        \end{gather}
        $(1)$ and $(2)$ together would imply $k|p$ which is impossible since $p$
        is a prime
    \item $|G|^{p-1} = s + kp$ and $p|s$ so $\exists x\in G$ such that $x^p = 1$
\end{enumerate}
\begin{theorem}{Sylow's Theorem}{}
    If $G$ is a finite group of order $p^\alpha m$ where $p$ is a prime and
    $p\nmid m$, then $G$ has a subgroup of order $p^\alpha$.
\end{theorem}
\begin{proposition}{}{}
    If $H$ and $K$ are finite subgroups of a group then,
    \begin{gather*}
        |HK| = \frac{|H||K|}{|H\cap K|}
    \end{gather*}
\end{proposition}{}{}
\textbf{Proof sketch}: $|HK|$ = number of distinct left cosets of $H$ in $K$
$\times$ size of each coset = $\frac{|H|}{|H\cap K|}\times |K|$
\begin{proposition}{}{}
    If $H$ and $K$ are subgroups of a group, $HK$ is a subgroup iff $HK = KH$.
\end{proposition}
\textbf{Proof sketch}:  For the forward implication, show that $HK\subseteq KH$
and $KH\subseteq HK$. For instance for arbitrary $h_1k_1\in HK$, $\exists
h_2k_2\in HK$ such that $h_1k_1 = (h_2k_2)^{-1} = k_2^{-1}h_2^{-1}\in KH$. For
the backward implication, observe that for some $a = h_1k_1$ and $b = h_2k_2$ in
$HK$, it suffices to show $ab^{-1} = h_1k_1k_2^{-1}h_2^{-1}\in HK$.

An interesting corollary is that if $ K \le H \le G$ and
$H\le N_G(K)$, then $HK$ is a subgroup of $G$ and in particular if
$K\trianglelefteq G$ then $HK\le G$, $\forall H\le G$.
\begin{definition}{}{}
    If $A$ is a subset of $N_G(K)$ (or $C_G(K)$), A is said to
    \textbf{normalize} (\textbf{centralize}) K.
\end{definition}
\begin{theorem}{First Isomorphism Theorem/\\Fundamental Theorem of Homomorphisms}{}
    If $\phi:G\rightarrow H$ is a homomorphism, then ker$\phi\trianglelefteq G$
    und $G/\text{ker}\phi\cong \phi(G)$.
\end{theorem}
\begin{theorem}{Second Isomorphism Theorem/\\Diamond Isomorphism Theorem}{}
    If $G$ is a group, $A$ and $B$ are subgroups of $G$ and $A\le N_G(B)$, then
    $AB$ is a subgroup of $G$, $B\trianglelefteq AB$, $A\cap B\trianglelefteq A$
    and $AB/B\cong A/A\cap B$.
\end{theorem}
Intuition: We have a subgroup $S$ and a normal subgroup $N$ of $G$. Now, we want
to quotient out $N$ from $S$ - however, we do not know if $N$ is also a subgroup
of $S$. Thus, there are two options:
\begin{enumerate}
    \item Quotient out $N$ from the smallest subgroup of $G$ containing both
        $S$\&$N$
    \item Quotient out the intersection from $S$
\end{enumerate}
\begin{theorem}{Third Isomorphism Theorem}{}
    If $G$ is a group, $A$ and $B$ are normal subgroups of $G$ such that $A\le
    B$, then $B/A\trianglelefteq G/A$ and 
    \begin{align*}
        (G/A)/(B/A) \cong G/B
    \end{align*}.
\end{theorem}
\begin{theorem}{Fourth Isomorphism Theorem/\\Lattice Isomorphism Theorem}{}
    If $G$ is a group and $N$ is a normal subgroup of $G$, then there is a
    bijection from the set of subgroups $A$ of $G$ which contain $N$ onto the
    set of subgroups $\overline A = A/N$ of $G/N$. In particular, every subgroup
    of $\overline G$ is of the form $A/N$ for some subgroup $A$ of $G$
    containing $N$. This bijection has the following properties, $\forall$
    subgroups $A, B$ containing $N$
    \begin{enumerate}
        \item $A\le B \iff \overline A\le\overline B$
        \item $A\le B\implies |B:A| = |\overline B:\overline A|$
        \item $\overline{\langle A, B\rangle} = \langle\overline A, \overline
            B\rangle$
        \item $\overline{A\cup B} = \overline A\cup\overline B$
        \item $A\trianglelefteq G$ iff $\overline A\trianglelefteq\overline G$
    \end{enumerate}
\end{theorem}
\begin{proposition}{}{}
    If $G$ is a finite abelian group and $p$ is a prime dividing $|G|$, then $G$
    contains an element of order $p$.
\end{proposition}
\textbf{Proof}: By induction on $G$
\begin{definition}{Simple group}{}
    A group $G$ is called \textbf{simple} if $|G|\ge 1$ and the only normal
    subgroups of $G$ are 1 and $G$.
\end{definition}
\begin{definition}{Composition series, factors}{}
    In a group $G$ a sequence of subgroups
    \begin{gather*}
        1 = N_0\le N_1\le\dots\le N_{k-1}\le N_k = G
    \end{gather*}
    is called a \textbf{composition series} if $N_i\trianglelefteq N_{i+1}$ and
    $N_{i+1}/N_i$ is a simple group, $0\le i\le k-1$ and the quotient groups
    $N_{i+1}/N_i$ are called \textbf{composition factors} of $G$.
\end{definition}
\begin{theorem}{Jordan-H\"older theorem}{}
    If $G$ is a finite group with $G\neq 1$ then
    \begin{enumerate}
        \item $G$ has a composition series
        \item The composition factors in a composition series are unique,
            namely, if $1 = N_0\le N_1\le\dots\le N_{r-1}\le N_r = G$ and $1 =
            M_0\le M_1\le\dots\le M_{s-1}\le M_s = G$ are two composition series
            for $G$, then $r = s$ and $\exists$ permutation $\pi$ of
            $\{1,\dots,n\}$ such that
            \begin{gather*}
                M_{\pi[(i)]}/M_{\pi[(i)] - 1}\cong N_i/N_{i - 1},\; 1\le i\le r
            \end{gather*}
    \end{enumerate}
\end{theorem}
\begin{center}
    \large{\textbf{The H\"older Program}}
    \begin{enumerate}
        \item Classify all finite simple groups.
        \item Find all ways of ``putting simple groups together'' to form other
            groups.
    \end{enumerate}
\end{center}
\begin{theorem}{Feit-Thompson theorem}{}
    If $G$ is a simple group of odd order, then $G\cong Z_p$ for some prime
    $p$.
\end{theorem}
\begin{quotation}
This proof takes 255 pages of hard mathematics
\end{quotation}
\begin{definition}{Solvable group}{}
    A group $G$ is \textbf{solvable} if $\exists$ a chain of subgroups
    \begin{gather*}
        1 = G_0\trianglelefteq G_1\dots\trianglelefteq G_s = G
    \end{gather*}
    such that $G_{i+1}/G_i$ is abelian $\forall i\in[0, s - 1]$
\end{definition}
\begin{theorem}{}{}
    A finite group $G$ is solvable iff for every divisor $n$ of $|G|$ such that
    $(n, \frac{|G|}{n}) = 1$, $G$ has a subgroup of order $n$.
\end{theorem}
\begin{proposition}{}{}
    If $N\trianglelefteq G$ such that $N$ and $G/N$ are solvable, the $G$ is
    solvable.
\end{proposition}
\begin{definition}{Transportation}{}
    A 2-cycle is called a \textbf{transportation}.
\end{definition}
\begin{definition}{Sign, Even/Odd permutation}{}
    \begin{enumerate}
        \item $\epsilon(\sigma)$ is called the \textbf{sign} of $\sigma$
        \item $\sigma$ is called an \textbf{even permutation} if
            $\epsilon(\sigma) = 1$ and \textbf{odd permutation} if
            $\epsilon(\sigma) = -1$.
    \end{enumerate}
\end{definition}
\begin{proposition}{}{}
    The map $\epsilon:S_n\rightarrow\{\pm 1\}$ is a homomorphism.
\end{proposition}
\begin{proposition}{}{}
    Transportations are all odd permutations and $\epsilon$ is a surjective
    homomorphism.
\end{proposition}
\begin{definition}{Alternating group}{}
    The \textbf{alternating group of degree $n$}, denoted by $A_n$, is the
    kernel of the homomorphism $\epsilon$
\end{definition}
\begin{proposition}{}{}
    The permutation $\sigma$ is odd iff the number of cycles of even length in
    its decomposition is odd
\end{proposition}
\begin{definition}{}{}
    For any group $G$,
    \begin{gather*}
        N = \langle x^{-1}y^{-1}xy | x, y\in G\rangle
    \end{gather*}
    is called the \textbf{commutator subgroup} of $G$.
\end{definition}

\chapter{Group Actions}

\begin{proposition}{}{}
    For any group $G$ and any nonempty set $A$ there is a bijection between the
    actions of $G$ on $A$ and the Homomorphisms of $G$ into $S_A$
\end{proposition}

\begin{proposition}{}{}
    If $G$ is a group acting on a nonempty set $A$ then the relation defined by
    \begin{align*}
        a\sim b\textrm{ iff } a = g.b,\textrm{ for some } g\in G
    \end{align*}
    is an equivalence relation. For each $a\in A$ the number of elements in the
    equivalence class containing $a$ is $|G:G_a|$, the index of the stabilizer
    of $a$
\end{proposition}
\textbf{Proof}: Observe that for each $a\in A$, the sets $G/$Stab$(a)$ and
Orbit$(a)$ are isomorphic.

\begin{definition}{}{}
    If $G$ is a group acting on a nonempty set $A$
    \begin{enumerate}
        \item The equivalence class $\{g.a | g\in G\}$ is called the
            \textbf{orbit} of $G$ containing $a$
        \item The action of $G$ on $A$ is called \textbf{transitive} if there
            is only one orbit, i.e., $\forall a, b\in A, \exists g\in G$ such
            that $a = g.b$
    \end{enumerate}
\end{definition}

\begin{theorem}{}{}
    If $G$ acts on its subgroup $H$ by left multiplication on its cosets, then
    \begin{enumerate}
        \item This action is transitive
        \item The stabilizer is $H$
        \item The kernel is $\cap_{x\in G} xHx^{-1}$ and is the largest normal
            subgroup of $G$ contained in $H$
    \end{enumerate}
\end{theorem}
\textbf{Proof:} $\cap_{x\in G} xHx^{-1}$ is a normal subgroup and every other
normal subgroup contained in $H$ is also a subgroup of $\cap_{x\in G} xHx^{-1}$.

\begin{theorem}{Cayley's theorem}{}
    Every group is isomorphic to a subgroup of some symmetric group. If $|G| =
    n$ then $G$ is isomorphic to some subgroup of $S_n$
\end{theorem}
\textbf{Proof:} Consider the group action by $G$ on its identity subgroup $\{e\}$
by coset multiplication. $G$ is isomorphic to its image in $S_n$, by the First
Isomorphism Theorem. The isomorphism is given by $T:G\rightarrow S_G$ where
$T(g) = \sigma_g$

\begin{theorem}{}{}
    If $G$ is a finite group of order $n$ and $p$ is the smallest prime dividing
    $n$, then any subgroup of index $p$ is normal
\end{theorem}
If $\phi:G\rightarrow S_p$, then $|G/ker(\phi)|$ divides $|G|$ and also $|S_p|$.
Therefore, it must divide $(|G|, p!) = p$ making it of order either 1 or $p$,
both of which imply normality.

\begin{proposition}{}{}
    Number of conjugates of a subset $S$ of a group $G$ is $|G:N_G(S)|$. In
    particular, the number of conjugates of an element $s$ is $|G:C_G(s)|$.
\end{proposition}

\begin{theorem}{Class Equation}{}
    If $G$ is a finite group and $g_1\dots g_r$ are the representatives of the
    distinct conjugacy classes not contained in $Z(G)$ then
    \begin{gather*}
        |G| = |Z(G)| + \sum_{i = 1}^r |G : C_G(g_i)|
    \end{gather*}
\end{theorem}

\begin{theorem}{}{}
    If $p$ is a prime and $|P| = p^\alpha$ for some $\alpha\geq 1$, then $P$ has
    a nontrivial center, $Z(P)\neq 1$
\end{theorem}
Observe that this implies if $|P| = p^2$ for some prime $p$ then $P$ is abelian
(specifically, either $Z_p\times Z_p$ or $Z_{p^2}$). Proof idea : use class
equation.

\begin{proposition}{}{}
    If $|P| = p^2$ for some prime $p$, then $P$ is abelian (and isomorphic to
    $Z_{p\times p}/Z_{p^2}$)
\end{proposition}
Proof sketch: By previous proposition, it has a nontrivial center making
$P/Z(P)$ cyclic implying $P$ is abelian. If $x$ is a nonidentity element in $P$
then it has order $p$ (if $p^2$ then we are done) and let $y\in P \setminus
\langle x\rangle$. Then since $|\langle x, y\rangle | > |\langle x\rangle |$,
the former must be $P$ itself and since each $x$ and $y$ have order $p$, we have
$\langle x\rangle \times \langle y\rangle\cong Z_{p\times p}$.

\begin{proposition}{}{}
    Conjugations in $S_n$ are just relabelling, i.e. if $\sigma = (a_1\dots
    a_n)$ then for any $\tau$,
    \begin{gather*}
        \tau\sigma\tau^{-1} = (\tau(a_1)\dots\tau(a_n))
    \end{gather*}
\end{proposition}

\begin{definition}{Cycle type}{}
    If $\sigma\in S_n$ is the product of disjoint cycles of lengths
    $n_1,\dots,n_r$ in nondecreasing order, then the integers $n_1,\dots,n_r$
    are called the \textbf{cycle type} of $\sigma$.
\end{definition}

\begin{proposition}{}{}
    Two elements of $S_n$ are conjugates in $S_n$ iff they have the same cycle
    type.
\end{proposition}
\textbf{Proof:} For the forward proof, observe that conjugate permutations have
the same cycle type as conjugation is just relabelling. For the backward proof,
observe that ordering the cycles in nondecreasing length gives us a permutation
mapping elements at corresponding positions in the two decompositions.

Useful observation: Normal subgroups are union of conjugacy classes

\begin{definition}{}{}
    The set of all isomorphisms from a group $G$ to itself is called the
    automorphism group, denoted by Aut$(G)$
\end{definition}

\begin{proposition}{}{}
    If $H\trianglelefteq G$ then $G$ acts by conjugation on $H$ as automorphisms
    of $H$, this action is defined by
    \begin{gather*}
        g.H\mapsto gHg^{-1}
    \end{gather*}
    Furthermore, the permutation representation afforded by this action is a
    Homomorphism of $G$ into Aut($H$) with kernel $C_G(H)$ and $G/C_G(H)$ is
    isomorphic to a subgroup of Aut($H$)
\end{proposition}
A few corollaries:
\begin{enumerate}
    \item $K\leq G\implies\forall g\in G, K\cong gKg^{-1}$
    \item For any subgroup $H\leq G,$ the quotient group $N_G(H)/C_G(H)$ is
        isomorphic to some subgroup of Aut$(G)$.
\end{enumerate}

\begin{definition}{}{}
    If $G$ is a group and $g\in G$, conjugation by $g$ is called an
    \textbf{inner automorphism} of $G$ and the subgroup of Aut$(G)$ consisting
    of all inner automorphisms is denoted by Inn$(G) (= N_G(H)/C_G(H))$
\end{definition}


\begin{definition}{}{}
    A subgroup $H$ of $G$ is called \textbf{characteristic} in $G$, denoted $H$
    char $G$, if every automorphism of $G$ maps $H$ onto itself -
    $\forall\sigma\in$ Aut$(G)$, $\sigma(H) = H$
\end{definition}

\begin{proposition}{}{}
    The automorphism group of the cyclic group of order $n$ is isomorphic to
    $(\mathbb{Z}/\mathbb{Z})^\times$, an abelian group of order $\phi(n)$
\end{proposition}

\begin{proposition}{}{}
    \begin{enumerate}
        \item If $p$ is an odd prime and $n\in\mathbb{Z}^+$ then the
            automorphism group of the cyclic group of order $p$ is cyclic of
            order $p-1$ and of order $p^n$ is $p^{n-1}(p-1)$
        \item $\forall n\geq 3$, Aut$(Z(2^n))\sim Z_2\times Z_{2^{n-2}}$ and in
            particular is not cyclic but has a cyclic subgroup of index 2
        \item 
    \end{enumerate}
\end{proposition}

\begin{definition}{}{}
    If $G$ is a group and $p$ a prime
    \begin{enumerate}
        \item A group of order $p^\alpha,\forall\alpha\geq 0$ is called a
            \textbf{p-group}. Subgroups of $G$ which are p-groups are called
            \textbf{p-subgroups}
        \item If $G$ is of order $p^\alpha m$ such that $p\nmid m$, then a
            subgroup of order $p^\alpha$ is called a \textbf{Sylow p-subgroup}
            of $G$
        \item The set of Sylow $p$-subgroups of $G$ will be denoted by
            $Syl_p(G)$ and the number of Sylow $p$-subgroups by $n_p(G)$
    \end{enumerate}
\end{definition}

\textbf{Lemma:} If $P\in Syl_p(G)$ and $Q$ is any $p$-subgroup of $G$, then
\begin{gather*}
Q\cap N_G(P) = Q\cap P
\end{gather*}
\textbf{Proof:} Right to left inclusion works as $P\leq N_G(P)$ and left to
right as $P(N_G(P)\cap Q)$ is a $p$-subgroup containing $P$, but $P$ has largest
possible order, so $Q\cap N_G(P)\leq P$.

\begin{theorem}{Sylow's Theorem}{}
    If $G$ is a group such that $|G| = p^\alpha m$ and $p\nmid m$ then
    \begin{enumerate}
        \item $Syl_p(G)\neq\phi$
        \item If $P$ is a Sylow $p$-subgroup and $Q$ any $p$-subgroup of $G$
            then $\exists g\in G$ such that $Q\leq gPg^{-1}$ ($Q$ is contained
            in some conjugate of $P$). In particular, any two Sylow
            $p$-subgroups of $G$ are conjugate in $G$.
        \item The number of Sylow $p$-subgroups of $G$ is of the form $1+kp$,
            i.e.
            \begin{gather*}
                n_p\equiv 1(mod\; p)
            \end{gather*}
            Furthermore, $n_p$ is the index in $G$ of the normalizer $N_G(P)$
            for any Sylow $p$-subgroup $P$, hence $n_p$ divides $m$
    \end{enumerate}
\end{theorem}
\textbf{Proof sketch:}

\chapter{Introduction to Rings}
\begin{definition}{Ring}{}
    A \textbf{ring} $R$ is a set with two binary operations $+$ and $\times$
    such that
    \begin{enumerate}
        \item $(R, +)$ is an abelian group
        \item $\times$ is associative : $(a\times b)\times c = a\times (b\times
            c)$
        \item The following distributive laws hold
            \begin{enumerate}
                \item $(a+b)\times c = (a\times c) + (b\times c)$
                \item $a\times (b+c) = (a\times b) + (a\times c)$
            \end{enumerate}
    \end{enumerate}
\end{definition}

A ring is commutative if multiplication is commutative

A ring is said to have an identity if $\exists 1\in R$ such that
\begin{gather*}
    1\times a = a\times 1 = a, \forall a\in R
\end{gather*}

\begin{definition}{Division ring}{}
    A ring $R$ with an identity element 1 ($\neq 0$) is called a
    \textbf{division ring/skew field} if inverses exist for all nonzero elements
    such that $ab=ba=1$
\end{definition}

A commutative division ring is called a \textbf{field}

\begin{proposition}{}{}
    If $R$ is a ring then
    \begin{enumerate}
        \item $0a=a0=0, \forall a\in R$
        \item $(-a)b=a(-b)=-(ab), \forall a,b\in R$
        \item $(-a)(-b) = ab,\forall a,b\in R$
        \item If $R$ has identity 1, then it is unique and $-a = (-1)a$
    \end{enumerate}
\end{proposition}

\begin{definition}{Zero divisor, unit}{}
    \begin{enumerate}
        \item A nonzero element $a\in R$ is called a \textbf{zero divisor} if
        $\exists b\neq 0\in R$ such that $ab=0$ or $ba=0$
        \item If $R$ has an identity $1\neq 0$, then $u\in R$ is called
        \textbf{unit} in $R$ if $\exists v\in R$ such that $uv=vu=1$. Set of
        units is denoted by $R^\times$
    \end{enumerate}
\end{definition}
 Note:
\begin{enumerate}
    \item The units in a ring $R$ form a group under multiplication
    \item A zero divisor can never be a unit
\end{enumerate}

\begin{definition}{Integral domain}{}
    A commutative ring with identity $1\neq 0$ is called an \textbf{integral
    domain} if it has no zero divisors
\end{definition}

\begin{proposition}{}{}
    If $a,b,c\in R$ and $a$ not a zero divisor, then $ab=ac\implies a = 0 \lor b =
    c$, equivalent $ab=ac\implies a = 0\lor b = c$
\end{proposition}

Note that any finite integral domain is a field since for any nonzero $a\in R$.
$aR$ is a bijective map

\begin{definition}{Subring}{}
    A \textbf{subring} of a ring $R$ is a subgroup of $R$ closed under
    multiplication
\end{definition}

\begin{definition}{Ring homomorphism, kernel}{}
    If $R$ and $S$ are rings then
    \begin{enumerate}
        \item A ring homomorphism is a map $\phi:R\rightarrow S$ such that
        \begin{enumerate}
            \item $\phi(a+b) = \phi(a) + \phi(b)$
            \item $\phi(ab) = \phi(a)\phi(b)$
        \end{enumerate}
        \item The kernel of $\phi$, denoted $ker\phi$ is the set of $r\in R$
            such that $\phi(r) = 0_S$
        \item A bijective ring homomorphism is called an isomorphism
    \end{enumerate}
\end{definition}

\begin{proposition}{}{}
    If $R$ and $S$ are two rings and $\phi:R\rightarrow S$ is a homomorphism,
    then
    \begin{enumerate}
        \item The image of $\phi$, $\phi(R)$ is a subring of $S$
        \item The kernel of $\phi$, $ker\phi$ is a subring of $R$
    \end{enumerate}
\end{proposition}

\begin{definition}
    If $R$ is a ring, $I\subseteq R$ and $r\in R$
    \begin{enumerate}
        \item $rI = \{ra|a\in I\}$ and $Ir = \{ar|a\in I\}$
        \item $I$ is a left ideal if
            \begin{enumerate}
                \item $I$ is a subring of $R$
                \item $I$ is a closed under left multiplication by elements of $R$
            \end{enumerate}
        \item $I$ is a right ideal if
            \begin{enumerate}
                \item $I$ is a subring of $R$
                \item $I$ is a closed under right multiplication by elements of $R$
            \end{enumerate}
        \item If $I$ is both a left and right ideal, then it is called a
            two-sided ideal
    \end{enumerate}
\end{definition}

\begin{proposition}{}{}
    If $I$ is an ideal of the ring $R$ then the additive quotient group $R/I$ is
    a ring, called the \textbf{quotient right} under the following binary
    operations
    \begin{gather*}
        (r + I) + (s + I) = (r + s) + I\\
        (r + I) \times (s + I) = (rs) + I\\
    \end{gather*}
\end{proposition}

\begin{theorem}{First Isomorphism Theorem for Rings}{}
    If $\phi:R\rightarrow S$ is a homomorphism of rings then
    \begin{enumerate}
        \item $ker\phi$ is an ideal of $R$
        \item $\phi(R)$ is isomorphic to a subring of $S$
        \item $R/ker\phi$ is isomorphic to $\phi(R)$
    \end{enumerate}
\end{theorem}

\begin{definition}{}{}
    If $I$ is any ideal of the ring $R$, then the map
    \begin{gather*}
        \phi:R\rightarrow S, r\rightarrow r+I
    \end{gather*}
    is a surjective ring homomorphism, called the natural projection of $R$ onto
    $R/I$
\end{definition}

\begin{theorem}{Second Isomorphism Theorem for Rings}{}
    If $S$ is a subring and $I$ is an ideal of a ring $R$ then $\{s + i|s\in S,
    i\in I\}$ is a subring, $S\cap I$ is an ideal of $S$ and $(S+I)/I\cong
    S/(S\cap I)$
\end{theorem}

\begin{theorem}{Third Isomorphism Theorem for Rings}{}
    If $I$ and $J$ are ideals of the ring $R$ such that $I\subseteq J$ then
    $J/I$ is an ideal of $R/I$ and $(R/I)/(J/I)\cong R/J$
\end{theorem}

\begin{theorem}{Fourth Isomorphism Theorem for Rings}{}
    If $I$ is an ideal of a ring $R$ then the correspondence $A\leftrightarrow
    R/I$ is an inclusion preserving bijection between set of subrings $A$ of $R$
    containing $I$ and set of subrings $R/I$
\end{theorem}

\begin{definition}{}{}
    Let $A$ be any subset of the ring $R$
    \begin{enumerate}
        \item The smallest ideal of $R$ containing $A$ is called the ideal
            generated by $A$, denoted by $(A)$
        \item $RA$ denotes the set of all finite sums of elements of the form
            $ra$ (where $r\in R$) i.e.,
            \begin{gather*}
                RA = \{a_1r_1+\dots+a_nr_n | a_i\in A, r_i\in R,
                n\in\mathbb{Z}^+\}
            \end{gather*}
        \item An ideal generated by a single element is called a
            \textbf{principal ideal}
        \item An ideal generated by a finite set is called a \textbf{finitely
            generated ideal}
    \end{enumerate}
\end{definition}

\begin{proposition}{}{}
    If $I$ is an ideal of the ring $R$
    \begin{enumerate}
        \item $I = R\iff I$ has a unit
        \item If $R$ is commutative, then $R$ is a field iff its only ideals are
            $0$ and $R$
    \end{enumerate}
\end{proposition}

\textbf{Corollary}: If $R$ is a field then any nonzero ring homomorphism from
$R$ into another ring is an injection

\begin{definition}{}{}
    An ideal $M$ in an arbitrary ring $S$ is called \textbf{maximal ideal} if
    $M\neq S$ and the only ideals containing $M$ are $M$ and $S$
\end{definition}

\begin{proposition}{}{}
    In a ring with identity every proper ideal is contained in a maximal ideal
\end{proposition}

\begin{proposition}{}{}
    If $R$ is commutative, then an ideal $M$ is maximal iff the quotient ring
    $R/M$ is a field
\end{proposition}

\begin{definition}{}{}
    If $R$ is commutative, an ideal $P$ is called a \textbf{prime ideal} if
    $P\neq R$ and whenever the product $ab$ of two elements $a, b\in R$ is in
    $P$, then at least one of $a$ and $b$ is in $P$
\end{definition}

\begin{proposition}{}{}
    If $R$ is commutative, then an ideal $P$ is prime in $R$ iff the quotient
    ring $R/P$ is an integral domain
\end{proposition}

\textbf{Corollary}: If $R$ is commutative, every maximal ideal of $R$ is prime

\end{document}
