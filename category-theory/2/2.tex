\documentclass[titlepage, 12pt]{article}
\usepackage[parfill]{parskip}
\usepackage{listings}

\begin{document}

\title{Category Theory Unit 2}

\author{Bharathi Ramana Joshi}

\date{\today}

\maketitle

\newpage

\section{Book}
\begin{enumerate}
	\item Monkeys banging at the keyboard:
		\begin{itemize}
			\item Machine language $\Rightarrow$ any combination produced should be run
			\item High level language $\Rightarrow$ lexical and grammatical errors detected
			\item Type checking further removes errors
			\item Question: Make monkeys happy or produce correct programs?
		\end{itemize}
	\item Intuitively, types are (finite or infinite) sets of values.
	\item A category of sets, \textbf{Set} has objects as sets and morphisms
		(arrows) as functions
	\item To overcome infinite recursion, a new type $\perp$ (bottom) is introduced
	\item \textbf{Operational semantics} describes the mechanics of program execution in
		terms of a formalized idealized interpreter
	\item \textbf{Denotational semantics} gives every programming construct its
		mathematical interpretation - proving a property of a program is same as
		proving a mathematical theorem
	\item Functions that always produce the same result given the same input and
		have no side effects are called \textbf{pure functions}
	\item The type corresponding to an empty set is \lstinline{Void} - it's a
		type that's not inhabited by any values
	\item The type corresponding to a singleton set is \lstinline{()}
	\item Functions from unit to any type $A$ are in one-to-one correspondence
		with the elements of that set $A$
	\item The type corresponding to a two element set is \lstinline{Bool}
\end{enumerate}

\end{document}

