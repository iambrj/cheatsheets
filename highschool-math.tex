\documentclass[titlepage, 12pt]{book}
\usepackage[parfill]{parskip}
\usepackage{amsmath}
\usepackage{xcolor}
\usepackage{amsfonts}
\usepackage{setspace}
\usepackage{hyperref}
\usepackage{tcolorbox}
\tcbuselibrary{theorems}

\hypersetup{
    colorlinks=true,
    linkcolor=blue,
    filecolor=magenta,
    urlcolor=blue,
}

\newtcbtheorem[]{definition}{Definition}%
{colback=magenta!5,colframe=magenta!100!black,fonttitle=\bfseries}{th}

\newtcbtheorem[]{proposition}{Proposition}%
{colback=cyan!5,colframe=cyan!100!black,fonttitle=\bfseries}{th}

\newtcbtheorem[]{theorem}{Theorem}%
{colback=orange!5,colframe=orange!100!black,fonttitle=\bfseries}{th}

\begin{document}

\title{Notes on high school mathematics}
\author{Bharathi Ramana Joshi\\\url{https://github.com/iambrj/notes}}
\maketitle

\tableofcontents

\chapter*{Preface}
\addcontentsline{toc}{chapter}{Preface}

I had learnt a lot of useful (and \textit{fun}!) mathematics in my high school,
but all my notes back then were handwritten. This document is an attempt at
digitalizing them since I still find some of that math useful and having a
digital document makes it easier to revisit material.

\chapter{Functions}
\begin{definition}{Cartesian product}{}
    If $A$ and $B$ are two sets, then their \textit{Cartesian product} is
    defined as the set $\{(a, b)\mid a\in A,b\in B\}$ denoted by $A\times B$
\end{definition}

\begin{definition}{Relation}{}
    If $A$ and $B$ are two sets, then any subset of $A\times B$ is called a
    \textit{relation} from $A$ to $B$
\end{definition}

\begin{definition}{Function}{}
    If $A$ and $B$ are two sets, then function $f$ from $A$ to $B$ is a relation
    from $A$ to $B$ such that $\forall a\in A,\exists b\in B$ such that $(a,
    b)\in f$. It is denoted by $f:A\rightarrow B$, the set $A$ is called the
    \textit{domain} and $b$ is called the \textit{co-domain} of $f$
\end{definition}

\begin{definition}{Image, Pre-image}{}
    If $f:A\rightarrow B$ is a function such that $f(a) = b$, then $b$ is called
    the \textbf{image} of $a$ under $f$ and $a$ is called the pre-image of $b$
    under $f$
\end{definition}

\begin{definition}{Range}{}
    If $f:A\rightarrow B$ is a function then $f(A)$, the set of all images is
    called the range of $f$
\end{definition}

\begin{definition}{Injection}{}
    A function $f:A\rightarrow B$ is called an
    \textbf{injection}/\textbf{one-to-one} if distinct elements of $A$ have
    distinct images in $B$
\end{definition}

\begin{definition}{Surjection}{}
    A function $f:A\rightarrow B$ is called an
    \textbf{surjection}/\textbf{onto} if range is equal to the co-domain
\end{definition}

\begin{definition}{Bijection}{}
    A function $f:A\rightarrow B$ is called an \textbf{bijection} if it is both
    into and onto
\end{definition}

\begin{definition}{Function equality}{}
    Two functions $f$ and $g$ are equal, iff
    \begin{enumerate}
        \item Both have same domains
        \item Images of all elements from the domain are the same, i.e. $f(x) =
            g(x), \forall x\in $ domain
    \end{enumerate}
\end{definition}

\begin{definition}{Inverse function}{}
    If $f:A\rightarrow B$ is a bijection, then the relation $f^{-1} = \{(b,
    a)\mid f(a) = b\}$ is defined as the inverse function of $f$
\end{definition}

\begin{definition}{Function composition}{}
    If $f : A\rightarrow B$ and $g : B\rightarrow C$ are two functions, then
    their composition function $f\circ g$ is defined as the relation $\{(a,
    g(f(a)))\mid a\in A\}$
\end{definition}

\chapter{Matrices}

\begin{definition}{Matrix}{}
    An ordered rectangular array of elements is called a \textbf{matrix}
\end{definition}

\begin{definition}{Order of a Matrix}{}
    A matrix having $m$ rows and $n$ columns is said to be of order $m\times n$,
    read a s$m$ cross $n$ or $m$ by $n$
\end{definition}

Some common matrices
\begin{enumerate}
    \item\textbf{Square matrix} : $m = n$
    \item\textbf{Diagonal matrix} : All non-diagonal elements of a square matrix
        are zeroes
    \item\textbf{Scalar matrix} : Diagonal matrix where all diagonal elements
        are equal
    \item\textbf{Unity/Identity matrix} : Diagonal matrix where all diagonal
        elements are 1
    \item\textbf{Null/Zero matrix} : All elements are zeroes
    \item\textbf{Row/Column matrix} : Matrix with only single row/column
    \item\textbf{Triangular matrix} : All elements below/above (lower/upper) are
        zeroes
\end{enumerate}

\begin{definition}{Equality of matrices}{}
    Two matrices are equal if they have same order and corresponding elements
    are equal
\end{definition}

\begin{definition}{Sum of two matrices}{}
    If two matrices are of equal order then their sum matrix is defined as sum
    of corresponding elements
\end{definition}

\begin{definition}{Scalar multiple of a matrix}{}
    The scalar multiple of a matrix is defined as the matrix obtained by
    multiplying each element by fixed scalar
\end{definition}

\begin{definition}{Product of two matrices}{}
    If $A = [a_{ik}]_{m\times n}$ and $B = [b_{kj}]_{n\times p}$ are two
    matrices, then their product $C = [c_{ij}]_{m\times p}$ is defined as
    $c_{ij} = \sum_{k = 1}^n a_{ik}b_{kj}$
\end{definition}

\begin{definition}{Transpose of a matrix}{}
\end{definition}
\end{document}

