\documentclass[titlepage, 12pt]{article}
\usepackage[parfill]{parskip}
\renewcommand{\familydefault}{\sfdefault}

\begin{document}

\title{Notes on PFPL}

\author{Bharathi Ramana Joshi}

\date{\today}

\maketitle

\tableofcontents

\newpage

\section{Syntactic Objects}

\begin{enumerate}

	\item Surface/Concrete syntax : concerned with how phrases are entered and
		displayed on a computer

	\item Structural/Abstract Syntax : concerned with the structure of phrases,
		specifically how they are composed from other phrases

		\begin{enumerate}
			
			\item Abstract Syntax Trees : nodes are operators that combine
				several phrases to form another phrase

			\item Abstract Binding Trees : enrich ASTs with the concepts of
				binding and scope (how identifiers are declared and how they are
				to be used)

		\end{enumerate}

\end{enumerate}

\subsection{Abstract Syntax Trees}

An ordered tree whose leaves are \textit{variables}, and whose interior nodes
are \textit{operators} whose arguments are its children.

\begin{enumerate}

	\item Variable : an unspecified/generic piece of syntax of specified sort

	\item Operator : a sort and an arity - a finite sequence of sorts specifying
		the number and sort of its arguments

\end{enumerate}

A variable is given meaning by substitution.

\subsubsection{Structural Induction}
If
$\{\mathcal{O}_s\}_{s\in \mathcal{S}}$ is a sort-indexed family of
		operators and
$\{\mathcal{X}_s\}_{s\in \mathcal{S}}$ is a sort-indexed family of
		variables
then, the family $\mathcal{A}[\mathcal{X}] = \{ \mathcal{A}[\mathcal{X}]_s
\}_{s\in \mathcal{S}}$ of ASTs of sort s is defined to be the smallest family
satisfying

\begin{enumerate}

	\item If $x\in\mathcal{X}_s$ then $x\in \mathcal{A}[\mathcal{X}]_s$

	\item If $o\in \mathcal{O}_s$ such that ar($o$) = ($s_1,\ldots,s_n$) and if
		$a_1\in\mathcal{A}[\mathcal{X}]_{s_1},\ldots,a_n\in\mathcal{A}[\mathcal{X}]_{s_n}$
		then $o(a_1;\ldots;a_n)\in\mathcal{A}[\mathcal{X}]_s$

\end{enumerate}

\textbf{Principle of Structural Induction}: To show that $\mathcal{P}(a)$ holds
for every $a\in\mathcal{A}[\mathcal{X}]$, it is sufficient to show that

\begin{enumerate}

	\item If $x\in\mathcal{X}_s$, then $\mathcal{P}_s(x)$

	\item If $o\in\mathcal{O}_s$ and $ar(o) = (s_1,\ldots,s_n)$, then if
		$\mathcal{P}_{s_1}(a_1),\ldots,\mathcal{P}_{s_n}(a_n)$ then
		$\mathcal{P}_s(o(a_1;\ldots;a_n))$

\end{enumerate}

Intuitively, show that the property holds in all the ways in which an AST may be
generated, under the assumption that it holds for each of its constituent ASTs.

\subsubsection{Substitution}

Variables are given meaning by substitution.

If $x$ is a variable of sort $s$, $a\in\mathcal{A}[\mathcal{X}, x]_{s\prime}$
and $b\in\mathcal{A}[\mathcal{X}]_s$, then
$[b/x]a\in\mathcal{A}[\mathcal{X}]_{s\prime}$ is defined to be the result of
substituting $b$ for every occurrence of $x$ in $a$. $a$ is called the
\textit{target} and $x$ is called the \textit{subject} of substitution. More
precisely,

\begin{enumerate}

	\item$[b/x]x = b$ and $[b/x]y = y$, if $x\neq y$

	\item$[b/x]o(a_1;\ldots;a_n) = o([b/x]a_1;\ldots;[b/x]a_n])$

\end{enumerate}

Theorem - Substitution on ASTs in well defined (proof idea - structural
induction)

\subsection{Abstract Binding Trees}



\end{document}

