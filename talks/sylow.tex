\documentclass{beamer}
\title{Sylow's theorem \& unsolvability of the quintic}
\author{Bharathi Ramana Joshi}
\begin{document}
\maketitle
\begin{frame}
    \frametitle{Motivation}
    \textbf{Lagrange's theorem} : $H\leq G\implies |H|\;\mid\;|G|$\\
    (Nice visual proof : \url{https://youtu.be/TCcSZEL_3CQ})\\
    \textbf{Example}: $Z_2\leq Z_4$ and $|Z_2|\;\mid\;|Z_4|$\\
    \vspace{20pt}
    \textbf{Converse}: $k\mid|G|\implies\exists H\leq G$ with $|H| = k$\\
    \textbf{Counterexample:} $A_4$ order 12, but no subgroup of order 6
\end{frame}

\begin{frame}
    \frametitle{Converse special case: Cauchy's theorem}
    \framesubtitle{\textit{Another Proof of Cauchy’s group theorem}, James H. McKay}
    What if prime $p\mid|G|$?
    Consider set of tuples
    \begin{gather*}
        T = \{(g_1,\dots,g_p):g_1\dots g_p = e\}
    \end{gather*}
    \begin{enumerate}
        \item $T$ partitioned into equivalence classes under cyclic permutations
        \item Each class has either $1$ or $p$ elements $\implies$ $|G|^{p-1} =
            k+pd$ (k = \# size-1 classes, d = \# size-p classes)
        \item $p|k\implies\exists x\in G$ such that $x^p = e$
    \end{enumerate}
\end{frame}

\begin{frame}
    \frametitle{Definitions}
    $G$ group, $p$ prime
    \begin{enumerate}
        \item $p$-subgroup: order $p^\alpha$
        \item Sylow $p$-subgroup: subgroup order $p^\alpha$, where group order
            $p^\alpha m (p\nmid m)$
        \item $Syl_p(G)$ set of Sylow $p$-subgroups
        \item $n_p(G) = |Syl_p(G)|$
    \end{enumerate}
\end{frame}

\begin{frame}
    \frametitle{Sylow's Theorems : Statement}
    \begin{enumerate}
        \item $n_p(G) \neq 0$
        \item $P$ Sylow $p$-subgroup and $Q$ any $p$-subgroup\\
            $\implies\exists g\in G$ such that $Q\leq gPg^{-1}$
        \item $n_p(G)\equiv 1(mod\; p)$
    \end{enumerate}
\end{frame}

\begin{frame}
    \frametitle{Sylow's Theorems : Application}
    \framesubtitle{Simplicity of $A_5$}
    $|A_5| = 60 = 2^2\times 3\times 5, n_5\in \{1,6\}, n_3\in \{1,4,10\}$\\
    Aiming for contradiction, $N\trianglelefteq A_5$. Cases;
    \begin{itemize}
        \item $5 | |N|$ or $3 | |N|$
        \item $|N| = 4\implies n_4 = 1, \textrm{ but } n_4 > 1$
        \item $|N| = 2\implies N = \langle (a_1\;a_2)\rangle$.\\
            But $(a_1\;a_2\;a_3)(a_1\;a_2)(a_1\;a_2\;a_3)^{-1} = (a_2\;a_3)\notin \langle (a_1\;a_2)\rangle$
    \end{itemize}
\end{frame}

\begin{frame}
    \frametitle{Sylow's Theorems : Proof outline}
    \begin{enumerate}
        \item Induction to prove existence
        \item Use count conjugates for 2\& 3
    \end{enumerate}
\end{frame}

\begin{frame}
    \frametitle{Sylow's Theorems : Existence proof}
    Cases
    \begin{enumerate}
        \item $p\mid |Z(G)|$
        \item $p\nmid |Z(G)|$
    \end{enumerate}
\end{frame}

\begin{frame}
    \frametitle{Existence proof : $p\mid |Z(G)|$}
    \begin{align*}
        \iff&\exists\; P\leq Z\ni |P| = p\\
        \iff& |G/P| = p^{\alpha - 1} m\\
        \iff& \exists |P'/P| = p^{\alpha - 1}\\
        \iff& |P'| = p^\alpha
    \end{align*}
\end{frame}

\begin{frame}
    \frametitle{Existence proof : $p\nmid |Z(G)|$}
    \begin{align*}
        &|G| = |Z| + \sum\frac{|G|}{|C_G(g_i)|}\\ \iff&\exists C_G(g_i)\ni |C_G(n_i)| = p^\alpha k
    \end{align*}
\end{frame}

\begin{frame}
    \frametitle{Lemma : Conjugate counting}
    $P$ Sylow $p$-subgroup and $Q$ any $p$-subgroup

    \begin{gather*}
        S = \{gPg^{-1} | g\in G\} = \{P_1,\dots,P_r\}
    \end{gather*}
    $Q$ acts on $S$ by conjugation
    \begin{gather*}
        S = O_1\cup\dots\cup O_s
    \end{gather*}
    Then
    \begin{gather*}
        |O_i| = |Q:N_Q(P_i)| = |Q:Q\cap N_G(P_i)| = |Q:Q\cap P_i|
    \end{gather*}
\end{frame}

\begin{frame}
    \frametitle{Lemma : Conjugate counting}
    \begin{gather*}
        |Q\cap N_G(P_i)| = |Q\cap P_i|\\
        \iff |P_i(Q\cap N_G(P_i))| = \frac{|P_i||Q\cap N_G(P_i)|}{|P_i\cap(Q\cap N_G(P_i))|}
    \end{gather*}
    For the particular case $Q = P (= P_1)$
    \begin{gather*}
        |O_1| = 1, |O_i| = |P_1:P_1\cap P_i| > 1
    \end{gather*}
    Thus \#conjugates
    \begin{gather*}
        |S| = |O_1| + (|O_2|\dots|O_s|)\equiv 1 (mod\;p)
    \end{gather*}
\end{frame}

\begin{frame}
    \frametitle{Sylow's Theorems : Containment \& Congruence to 1}
    Aiming for contradiction, let $Q$ not be contained in any conjugate.
    Then
    \begin{gather*}
        |O_i| = |Q:Q\cap P_i|
    \end{gather*}
    Thus $p$ divides \#orbits$\implies$ contradiction!\\
    \vspace{10mm}
    Since all Sylow $p$-subgroups are conjugates, $S = Syl_p(G)$
\end{frame}

\begin{frame}
    \frametitle{Exercises}
    \begin{enumerate}
    \item Write a program that given $n$, finds all permissible values of $n_p$
        for all groups $G$ of odd size $< n$ with $|Syl_p(G)|\neq 1$ for each
        prime divisor $p$ of group size.
        \item $P$ normal and $P\in Syl_p(G)\implies$
        \begin{enumerate}
            \item $|Syl_p(G)| = 1$
            \item $P$ characteristic in $G$
        \end{enumerate}
    \item $G$ simple and $|G| = 60\implies G\cong A5$
    \end{enumerate}
\end{frame}

\end{document}
