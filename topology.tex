\documentclass[titlepage, 12pt]{book}
\usepackage[parfill]{parskip}
\usepackage{amsmath}
\usepackage{xcolor}
\usepackage{amsfonts}
\usepackage{setspace}
\usepackage{hyperref}
\usepackage{tcolorbox}
\tcbuselibrary{theorems}

\hypersetup{
    colorlinks=true,
    linkcolor=blue,
    filecolor=magenta,
    urlcolor=blue,
}

\newtcbtheorem[]{definition}{Definition}%
{colback=magenta!5,colframe=magenta!100!black,fonttitle=\bfseries}{th}

\newtcbtheorem[]{proposition}{Proposition}%
{colback=cyan!5,colframe=cyan!100!black,fonttitle=\bfseries}{th}

\newtcbtheorem[]{theorem}{Theorem}%
{colback=orange!5,colframe=orange!100!black,fonttitle=\bfseries}{th}

\begin{document}

\begin{titlepage}

	\raggedleft

	\vspace*{\baselineskip}

	{Bharathi Ramana Joshi\\\url{https://github.com/iambrj/notes}}

	\vspace*{0.167\textheight}

    \textbf{\LARGE Notes on}\\[\baselineskip]

	\textbf{\textcolor{teal}{\huge James Munkres, Topology}}\\[\baselineskip]

    {\Large \textit{2nd Edition}}

	\vfill

	\vspace*{3\baselineskip}

\end{titlepage}

\newpage

\chapter{Topological Spaces \& Continuous Functions}

\begin{definition}{Topology}{}
 A topology on a set $X$ is a collection $\mathcal{T}$ of subsets of $X$ such
 that
 \begin{enumerate}
     \item $\phi$ and $X$ are in $\mathcal{T}$
     \item The union of the elements of any subcollection of $\mathcal{T}$ is in
         $\mathcal{T}$
     \item The intersection of the elements of any finite subcollection of
         $\mathcal{T}$ is in $\mathcal{T}$
 \end{enumerate}
\end{definition}

\begin{definition}{Open set}{}
    For a topology $(X, \mathcal{T})$, a subset $U\subseteq X$ is called
    \textbf{open} if $U\in\mathcal{T}$
\end{definition}

\textbf{Examples}
\begin{enumerate}
    \item\textbf{Discrete topology}: Collection of all subsets (power set)
    \item\textbf{Indiscrete/trivial topology}: Collection containing just the
        empty set and the entire set
    \item\textbf{Finite complement topology}: Collection of all subsets
        $U\subseteq X$ such that $X - U$ is either finite or all of $X$
    \item\textbf{Standard topology on }$\mathbb{R}$: collection of all open
        intervals
        \begin{gather*}
        \{(a, b)|a, b\in\mathbb{R}\}
        \end{gather*}
    \item\textbf{Lower limit topology on }$\mathbb{R}$: collection of all
        half-open intervals
        \begin{gather*}
            \mathbb{R}_l = \{[a, b)|a, b\in\mathbb{R}\}
        \end{gather*}
    \item\textbf{K-topology on }$\mathbb{R}$:
        In addition, to all open intervals $(a, b)$, \textbf{K-topology} has
        sets of the form
        \begin{gather*}
           \mathbb{R}_K = \{(a, b)|a, b\in\mathbb{R}\} - \{1/n | n\in\mathbb{Z}^+\}
        \end{gather*}
        Note that 5,6 are strictly finer than 4 but are not comparable with each
        other
\end{enumerate}

\begin{definition}{Finer/Coarser topology}{}
    If two topologies on $X$, $\mathcal{T}_1$ and $\mathcal{T}_2$ are such that
    $\mathcal{T}_1\supset\mathcal{T}_2$ then
    \begin{enumerate}
        \item $\mathcal{T}_1$ is called the \textbf{finer/larger/stronger} topology
        \item $\mathcal{T}_2$ is called the \textbf{coarser/smaller/weaker} topology
    \end{enumerate}
\end{definition}

\begin{definition}{Basis}{}
    A \textbf{basis} for a topology on a set $X$ is a collection $\mathcal{B}$
    of subsets of $X$ such that
    \begin{enumerate}
        \item $\forall x\in X$, there is a basis element $B$ containing $x$
            (i.e. union of everything gives $X$)
        \item If $x\in B_1\cap B_2$, then there is a basis element $B_3$ such
            that $x\in B_3\subset B_1\cap B_3$
    \end{enumerate}
    A subset $U\subset X$ is open if for every $x\in U$ there is a basis
    element $B$ such that $x\in B\subset U$
\end{definition}

Ex : Circular and rectangular regions on a plane from a basis for $\mathbb{R}^2$

\textbf{Lemma}: The topology generated by a basis is the collection of arbitrary
unions of its elements

\textbf{Lemma}: If $\mathcal{C}$ is a collection of open sets over a topological
space $X$ such that for every open set $U$ of $X$ and every $x\in U$ there is a
$C\in\mathcal{C}$ such that $x\in C\subseteq U$ then $\mathcal{C}$ is a basis
for the topology on $X$.

\textbf{Lemma}: If $\mathcal{B}$ and $\mathcal{B}'$ are bases for topologies
$\mathcal{T}$ and $\mathcal{T}'$ then the following are equivalent
\begin{enumerate}
    \item $\mathcal{T}'$ is finer than $\mathcal{T}$
    \item For each $x\in X$ and each $B\in\mathcal{B}$ containing $x$, there is
        a $B'\in\mathcal{B}'$ such that $x\in B'\subset B$ ($\forall x\in X$,
        $\forall \mathcal{B}\ni B\ni x\implies \exists \mathcal{B}'\ni
        B'\subset\mathcal{B}$)
\end{enumerate}

\begin{definition}{Subbasis}{}
    A \textbf{subbasis} $\mathcal{S}$ for a topology on $X$ is a collection of
    subsets whoswe union equals $X$. The topology generated by $\mathcal{S}$ is
    the collection of arbitrary unions of finite intersections of elements of
    $\mathcal{S}$
\end{definition}
Ex: all semi-infinite open intervals of the form $(-\infty, a)\cup (b, \infty)$
for arbitrary reals $a, b$ form a subbasis for the standard topology on
$\mathbb{R}$.

\begin{definition}{Order topology}{}
    Let $X$ be a set with more than one element and a total ordering. The
    collection $\mathcal{B}$ of subsets of
    \begin{enumerate}
        \item all open intervals $(a, b)$ in $X$
        \item all half open intervals $[a_0, b)$ where $a_0$ is the minimum
            element (if it exists)
        \item all half open intervals $(a, b_0]$ where $b_0$ is the maximum
            element (if it exists)
    \end{enumerate}
    form the basis for the \textbf{order topology}
\end{definition}
Ex: collection of open sets for standard topology, positive integers etc

\begin{definition}{Product topology}{}
    If $X$ and $Y$ are topological spaces, then the \textbf{product topology} on
    $X\times Y$ is the topology generated by the basis $\{U\times V |
    U\subset_{op} X, V\subset_{op} Y\}$
\end{definition}

\begin{theorem}{Basis of product topology}{}
    If $\mathcal{B}$ and $\mathcal{C}$ are basis for $X$ and $Y$, then
    $\mathcal{B}\times\mathcal{C}$ is a basis for $X\times Y$
\end{theorem}

\begin{theorem}{Subbasis for product topology}{}
    The collection
    \begin{align*}
        S = \{\pi_1^{-1}(U) | U\subset_{op}X\}\cup \{\pi_2^{-1}(V) |
        V\subset_{op} Y\}
    \end{align*}
    forms a subbasis for the product topology $X\times Y$
\end{theorem}
where $\pi_1^{-1}(U) = U\times Y (\subset_{op} X\times Y)$ and $\pi_2^{-1}(V) =
X\times V (\subset_{op} X\times Y)$, i.e. their intersection is $U\times V$

\begin{definition}{Subspace topology}{}
    Let $(X, \mathcal{T})$ be a topology and $Y\subset X$ then the collection
    \begin{align*}
        \mathcal{T}_Y = \{Y\cap U | U\in\mathcal{T}\}
    \end{align*}
    is a topology on $Y$, called the \textbf{subspace topology}
\end{definition}

\begin{theorem}{Subspace basis}{}
    If $\mathcal{B}$ is a basis for a topology on $X$, then the collection
    \begin{align*}
        \mathcal{B}_Y = \{B\cap Y | B\in\mathcal{B}\}
    \end{align*}
    is a basis for the subspace topology on $Y$
\end{theorem}

\begin{theorem}{Transitive openness in subspace topology}{}
    If $Y$ forms a subset topology on $X$ and $U\subset_{op}Y$ and
    $Y\subset_{op}X$ then $U\subset_{op} X$
\end{theorem}

\begin{theorem}{Subspace and product topology}{}
    If $A$ is a subspace of $X$ and $B$ a subspace of $Y$, then the product
    topology on $A\times B$ is same as the topology $A\times B$ inherits as a
    subspace of $X\times Y$
\end{theorem}

\textbf{Note}: the same need \textbf{not} hold for a subspace and order topology

\begin{definition}{Convex subset}{}
    A subset $Y$ of an ordered set $X$ is \textbf{convex} if for all points $a,
    b\in Y$ where $a < b$, the entire interval of points $(a, b)$ of $X$ are in
    $Y$
\end{definition}

For example,
\begin{enumerate}
    \item If $X = \mathbb{R}$ ordered by <, then any line (segment) is a convex subset.
    \item If $X = \mathbb{R}^2$ ordered by distance from origin, then any disc
        is a convex subset.
\end{enumerate}

\begin{theorem}{Subspace and order topology for convex sets}{}
    If $Y$ is a convex set, then the order topology it inherits from $X$ is same
    as the topology $Y$ inherits as a subspace of $X$
\end{theorem}
 
\begin{definition}{Closed set}{}
    A subset $A$ of a topological space $X$ is said to be \textbf{closed} if the
    set $X - A$ is open
\end{definition}
 
Ex: In the finite complement topology on a set $X$, the closed sets are $X$ and
all finite subsets of $X$; in the discrete topology on $X$, every set is open
and every set is closed.
 
Fun riddle - ``How is a set different from a door?"

\begin{theorem}{Properties of closed sets}{}
    If $X$ is a topological space then
    \begin{enumerate}
        \item $\phi$ and $X$ are closed
        \item Arbitrary intersections of closed sets are closed
        \item Finite unions of closed sets are closed
    \end{enumerate}
\end{theorem}

\begin{theorem}{Subspace and closedness}{}
    If $Y$ is a subspace of $X$ then a set $A$ is closed in $Y$ iff it is the
    intersection of a closed set of $X$ with $Y$
\end{theorem}

\begin{theorem}{Transitivity and closedness}{}
    If $Y$ is a subspace of $X$, $A$ is closed in $Y$, and $Y$ is closed in $X$,
    then $A$ is closed in $X$
\end{theorem}

\begin{definition}{Interior and closure}{}
    If $A$ is a subset of the topological space $X$ then then
    \textbf{\textit{interior}} of $A$ is defined as the union of all open sets
    contained in $A$ (denoted bty Int $A$) and the \textbf{\textit{closure}} of
    $A$ is defined as the intersection of all closed sets containing $A$
    (denoted by $\bar{A}$)
\end{definition}

Intuitively, the interior is the largest open set contained in $A$ and the
closure is the smallest closed set $A$. E.g.  $[0, 1]$ is the closure of $(0,
1)$ in the standard topology on $\mathbb{R}$, the interior and the surface of a
3D sphere are the closure of the sphere in 3D space etc.

\begin{theorem}{Subspace and closure}{}
    If $Y$ is a subspace of $X$ and $A$ a subset of $Y$, then the closure of $A$
    in $Y$ = $\bar{A}\cap Y$
\end{theorem}

\begin{theorem}{Elements in the closure}{}
    If $A$ is a subset of the topological space $X$, then
    \begin{enumerate}
        \item $x\in\bar{A}$ iff every open set $U$ containing $x$ intersects $A$
        \item $x\in\bar{A}$ iff every basis element $B$ containing $x$
            intersects $A$
    \end{enumerate}
\end{theorem}

\begin{definition}{Limit point}{}
    If $A$ is a subset of the topological space $X$ and $x\in X$, then $x$ is
    said to be a limit point of $A$ if every neighborhood of $x$ intersects $A$
    in some point other than $x$ itself. In other words, $x$ is a limit point of
    $A$ if it belongs to the closure of $A - \{x\}$. Note that the point may or
    may not lie in $A$ for this definition of a limit point.
\end{definition}

Note : "$U$ is a neighborhood of $x$" is short for "$U$ is an open set
containing $x$"

\begin{theorem}{Limit points and closure}{}
    If $A$ is the subset of the topological space $X$ and $A'$ the set of all
    limit points of $A$, then
    \begin{align*}
        \bar{A} = A\cup A'
    \end{align*}
\end{theorem}

\begin{definition}{Hausdorff space}{}
    A topological space $X$ is a Hausdorff space if $\forall x_1, x_2\in X$ that
    are distinct, there exist neighborhoods $U_1$ and $U_2$ of $x_1$ and $x_2$
    respectively that are disjoint.
\end{definition}
Motivation -- to restrict to spaces that do not have weird behaviour such as
sequences converging to multiple points and one-point sets are not closed. One
such weird space is $X = \{a, b, c\}$ and $T = \{\{\}, \{a, b, c\}, \{a, b\},
\{c, b\}, \{b\}\}$. Here the one point set $\{b\}$ is not closed and the
sequence $x_n = b$ converges to all three points $a, b, c$.

\begin{theorem}{Every finite point set in a Hausdorff space $X$ is closed}{}

\end{theorem}

\end{document}

