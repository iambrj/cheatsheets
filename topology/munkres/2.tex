\documentclass[titlepage, 12pt]{article}
\usepackage[parfill]{parskip}
\usepackage{amsmath}
\usepackage{xcolor}
\usepackage{amsfonts}
\usepackage{setspace}
\usepackage{hyperref}
\usepackage{tcolorbox}
\tcbuselibrary{theorems}

\hypersetup{
    colorlinks=true,
    linkcolor=blue,
    filecolor=magenta,
    urlcolor=blue,
}

\newtcbtheorem[]{definition}{Definition}%
{colback=magenta!5,colframe=magenta!100!black,fonttitle=\bfseries}{th}

\newtcbtheorem[]{proposition}{Proposition}%
{colback=cyan!5,colframe=cyan!100!black,fonttitle=\bfseries}{th}

\newtcbtheorem[]{theorem}{Theorem}%
{colback=orange!5,colframe=orange!100!black,fonttitle=\bfseries}{th}

\begin{document}

\begin{titlepage}

	\raggedleft

	\vspace*{\baselineskip}

	{Bharathi Ramana Joshi\\\url{https://github.com/iambrj/notes}}

	\vspace*{0.167\textheight}

    \textbf{\LARGE Notes on}\\[\baselineskip]

	\textbf{\textcolor{teal}{\huge Topological Spaces \& Continuous Functions}}\\[\baselineskip]

    {\Large \textit{James Munkres, Topology 2nd Ed}}

	\vfill

	\vspace*{3\baselineskip}

\end{titlepage}

\newpage

\begin{definition}{Topology}{}
 A topology on a set $X$ is a collection $\mathcal{T}$ of subsets of $X$ such
 that
 \begin{enumerate}
     \item $\phi$ and $X$ are in $\mathcal{T}$
     \item The union of the elements of any subcollection of $\mathcal{T}$ is in
         $\mathcal{T}$
     \item The intersection of the elements of any finite subcollection of
         $\mathcal{T}$ is in $\mathcal{T}$
 \end{enumerate}
\end{definition}

\begin{definition}{Open set}{}
    For a topology $(X, \mathcal{T})$, a subset $U\subseteq X$ is called
    \textbf{open} if $U\in\mathcal{T}$
\end{definition}

\textbf{Examples}
\begin{enumerate}
    \item\textbf{Discrete topology}: Collection of all subsets
    \item\textbf{Indiscrete/trivial topology}: Collection containing just the
        empty set and the entire set
    \item\textbf{Finite complement topology}: Collection of all subsets
        $U\subseteq X$ such that $X - U$ is either finite or all of $X$
    \item\textbf{Standard topology on }$\mathbb{R}$: collection of all open
        intervals
        \begin{gather*}
        \{(a, b)|a, b\in\mathbb{R}\}
        \end{gather*}
    \item\textbf{Lower limit topology on }$\mathbb{R}$: collection of all
        half-open intervals
        \begin{gather*}
        \{[a, b)|a, b\in\mathbb{R}\}
        \end{gather*}
    \item\textbf{K-topology on }$\mathbb{R}$:
        \begin{gather*}
            \{(a, b)|a, b\in\mathbb{R}\} - \{1/n | n\in\mathbb{Z}^+\}
        \end{gather*}
        Note that 5,6 are strictly finer than 4 but are not comparable with each
        other
\end{enumerate}

\begin{definition}{Finer/Coarser topology}{}
    If two topologies on $X$, $\mathcal{T}_1$ and $\mathcal{T}_2$ are such that
    $\mathcal{T}_1\supset\mathcal{T}_2$ then
    \begin{enumerate}
        \item $\mathcal{T}_1$ is called the \textbf{finer/larger/stronger} topology
        \item $\mathcal{T}_2$ is called the \textbf{coarser/smaller/weaker} topology
    \end{enumerate}
\end{definition}

\begin{definition}{Basis}{}
    A \textbf{basis} for a topology on a set $X$ is a collection $\mathcal{B}$
    of subsets of $X$ such that
    \begin{enumerate}
        \item $\forall x\in X$, there is a basis element $B$ containing $x$
        \item If $x\in B_1\cap B_2$, then there is a basis element $B_3$ such
            that $x\in B_3\subseteq B_1\cap B_3$
    \end{enumerate}
    A subset $U\subseteq X$ is open if for every $x\in U$ there is a basis
    element $B$ such that $x\in B\subseteq U$
\end{definition}

\textbf{Lemma}: The topology generated by a basis is the collection of arbitrary
unions of its elements

\textbf{Lemma}: If $\mathcal{C}$ is a collection of open sets over a topological
space $X$ such that for every open set $U$ of $X$ and every $x\in U$ there is a
$C\in\mathcal{C}$ such that $x\in C\subseteq U$ then $\mathcal{C}$ is a basis
for the topology on $X$.

\textbf{Lemma}: If $\mathcal{B}$ and $\mathcal{B}'$ are bases for topologies
$\mathcal{T}$ and $\mathcal{T}'$ then the following are equivalent
\begin{enumerate}
    \item $\mathcal{T}'$ is finer than $\mathcal{T}$
    \item For each $x\in X$ and each $B\in\mathcal{B}$ containing $x$, there is
        a $B'\in\mathcal{B}'$ such that $x\in B'\subset B$
\end{enumerate}

\begin{definition}{Subbasis}{}
    A \textbf{subbasis} $\mathcal{S}$ for a topology on $X$ is a collection of
    subsets whoswe union equals $X$. The topology generated by $\mathcal{S}$ is
    the collection of arbitrary unions of finite intersections of elements of
    $\mathcal{S}$
\end{definition}

\end{document}

