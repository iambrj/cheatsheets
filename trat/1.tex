\documentclass[titlepage, 12pt]{article}
\usepackage[parfill]{parskip}
\usepackage{amsmath}

\begin{document}

\title{Term Rewriting and All That\\Chapter 1 notes}

\author{Bharathi Ramana Joshi}

\date{Compiled: \today}

\newpage

\begin{itemize}

  \item First-order language: collection $S$ of symbols for relations, functions and
    constants, which, in combination with the symbols of elementary logic,
    single out certain combinations of symbols as sentences.

  \item Define addition using constant $0$ and successor function $s$
    \begin{gather*}
      x + 0 \approx x\\
      x + s(y) \approx s(x + y)
    \end{gather*}

  \item\textbf{Terms}: built from \textbf{variables}, \textbf{constant symbols}
    and \textbf{function symbols}.

  \item\textbf{Termination}: Is it always the case that after finitely many rule
    applications we reach an expression to which no more rules apply?

  \item\textbf{Confluence}: If there are different ways of applying rules to a
    given term leading to different terms $t_1$ and $t_2$, can $t_1$ and $t_2$
    can be joined, i.e. can we always find a common term $s$ that can be reached
    both from $t_1$ and $t_2$ by rule application?

  \item\textbf{Completion}: Can we always make a non-confluent system confluent
    by adding implied rules?

  \item\textbf{Word problem}: given a set of identities $E$ and two terms $s$
    and $t$, is it possible to transform the term $s$ into the term $t$, using
    the identities in $E$ as rewrite rules that can be applied in both
    directions?

\end{itemize}

\end{document}

